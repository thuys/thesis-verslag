\documentclass[master=cws,masteroption=gs]{kulemt}

\setup{title={Evaluatie van consistentie en beschikbaarheid in gedistribueerde database systemen},
  author={Thomas Uyttendaele},
  promotor={Prof.\,dr.\,ir.\ Wouter Joosen},
  assessor={dr.\,ir.\ Tias Guns, \\ Prof.\,dr.\,ir.\ Christophe Huygens},
  assistant={Dr.\ ir.\ Bart Vanbrabant \\ Dr. Bert Lagaisse}}
% De volgende \setup mag verwijderd worden als geen fiche gewenst is.
\setup{filingcard,
  translatedtitle={Evaluation of consistency and availability for distributed database systems},
  udc=681.3,
  shortabstract={In de database wereld zijn er momenteel twee grote categorieën van database systemen actief: de relationele en NoSQL systemen. De laatste categorie heeft uit een grote diversiteit aan datamodellen, performantie-eigenschappen en keuzes in het CAP theorema. \\
  Deze thesis focust op de uitwerking van een benchmarking tool om consistentie en beschikbaarheid in een gedistribueerde omgeving te testen. \\
  De beschikbaarheidstesten onderzoeken het gedrag van het gehele database systeem bij het in- en uitschakelen van een service, server of netwerk verbinding. De consistentietest bestudeert hoe het systeem omgaat met gelijktijdige schrijf- en leesbewerkingen op dezelfde dataeenheid. \\
  Deze benchmark onderzoek het gedrag van een database systeem in de praktijk en vergelijkt dit met de documentatie. Daarnaast worden verschillende database systemen met elkaar vergeleken. \\
  Drie verschillende database management systemen worden getest met behulp van de benchmarking tool, waarna de resultaten geanalyseerd worden. Op basis van deze resultaten komt verschillend gedrag tussen de systemen tot uiting zoals op een verschillende wijze omgaan met gelijktijdig lezen en schrijven van een data-element. \\
  De installatie en configuratie van de benchmarking tool en de reeds geteste database systemen is geautomatiseerd om te kunnen opstellen en uitvoeren zonder voorkennis van de specifieke systemen. Met een beperkte kennis en kost kan de tool ook uitgevoerd worden op andere database systemen. 
  }
 }
% Verwijder de "%" op de volgende lijn als je de kaft wil afdrukken
%\setup{coverpageonly}
% Verwijder de "%" op de volgende lijn als je enkel de eerste pagina's wil
% afdrukken en de rest bv. via Word aanmaken.
%\setup{frontpagesonly}

% Kies de fonts voor de gewone tekst, bv. Latin Modern
\setup{font=lm}

% Hier kun je dan nog andere pakketten laden of eigen definities voorzien

% Tenslotte wordt hyperref gebruikt voor pdf bestanden.
% Dit mag verwijderd worden voor de af te drukken versie.

\usepackage[hyphens]{url}
\usepackage[pdfusetitle,hidelinks, plainpages]{hyperref}
\usepackage[dutch]{babel}
\usepackage{lipsum} % for dummy text only
\usepackage{csquotes}
\usepackage[backend=bibtex]{biblatex}
%\usepackage[acronym]{glossaries}
\usepackage{eurosym}
\usepackage{subfigure}
\usepackage{multirow}
\usepackage{tikz}
\usepackage{listings}
\usepackage[font={small}]{caption}
\usepackage{pdfpages}

\usepackage{todonotes}
\usepackage{placeins}

\usepackage{tikz}
\usepackage{pgfplots}
\usepackage{verbatim}
\usepackage{color}
\usepackage{array}
\newcolumntype{L}[1]{>{\raggedright\let\newline\\\arraybackslash\hspace{0pt}}m{#1}}
\newcolumntype{C}[1]{>{\centering\let\newline\\\arraybackslash\hspace{0pt}}m{#1}}
\newcolumntype{R}[1]{>{\raggedleft\let\newline\\\arraybackslash\hspace{0pt}}m{#1}}

%\newglossaryentry{eventualconsistency}{
	name=eventuele consistentie,
	description= aaa
}

\newglossaryentry{nosql}{
	name=NoSQL,
	description= aaa
}

\newglossaryentry{horizontaalschaalbaar}{
	name=horizontaal schaalbaar,
	description= aaa
}

\newglossaryentry{yum}{
	name=yum,
	description= aaa
}

\newglossaryentry{apt-get}{
	name=apt-get,
	description= aaa
}

\newglossaryentry{CAP}{
	name=CAP,
	description= aaa
}

\newglossaryentry{IDP}{
	name=IDP,
	description= aaa
}
\makeglossaries
\bibliography{overige/referenties}
%%%%%%%
% Om wat tekst te genereren wordt hier het lipsum pakket gebruikt.
% Bij een echte masterproef heb je dit natuurlijk nooit nodig!
\IfFileExists{lipsum.sty}%
 {\usepackage{lipsum}\setlipsumdefault{11-13}}%
 {\newcommand{\lipsum}[1][11-13]{\par Hier komt wat tekst: lipsum ##1.\par}}
%%%%%%%
\definecolor{kuleuvenblue}{RGB}{220,221,222}
%\includeonly{hfdst-n}
\begin{document}

\setlength{\parindent}{0cm}
\setlength{\parskip}{\baselineskip}

\begin{preface}
Het afgelopen jaar heeft mijn thesis mij talrijke nieuwe ervaringen en uitdagingen gegeven. Het heeft mij kennis bijgebracht over de verschillende database systemen en de probleemwereld van het installeren, configureren, onderhouden en het gedrag van data systemen. 

Deze thesis is tot stand gekomen door de hulp en steun van verschillende mensen, ik neem graag even de tijd om deze te bedanken. 

Ik zou graag beginnen met mijn begeleiders Bart Vanbrabant en Bert Lagaisse voor hun permanente begeleiding en hulp bij het onderzoeken en schrijven van de thesis, klaar om met een fris oog en ander perspectief naar het werk te kijken. 

Mijn promotor Wouter Joosen zou ik graag willen bedanken voor het vertrouwen en de ondersteuning doorheen het jaar. 

Met mijn studiegenoot Arnaud Schoonjans heb ik vele vruchtbare gesprekken gehad over de mogelijke testmethodes en onze kennis over de verschillende database systemen kunnen delen. 

Tenslotte wil ik mijn familie, studiegenoten en vrienden bedanken voor hun steun in de thesis en het brengen van extra motivatie op de momenten dat ik het nodig had. 

  
\end{preface}

\tableofcontents*
\begin{abstract}
In de database wereld zijn er momenteel twee grote categorieën van database systemen actief: de relationele en NoSQL systemen. De laatste categorie heeft uit een grote diversiteit aan datamodellen, performantie-eigenschappen en keuzes in het CAP theorema. \\
Deze thesis focust op de uitwerking van een benchmarking tool om consistentie en beschikbaarheid in een gedistribueerde omgeving te testen. 
 
De beschikbaarheidstesten onderzoeken het gedrag van het gehele database systeem bij het in- en uitschakelen van een service, server of netwerk verbinding. De consistentietest bestudeert hoe het systeem omgaat met gelijktijdige schrijf- en leesbewerkingen op dezelfde dataeenheid. 

Deze benchmark onderzoek het gedrag van een database systeem in de praktijk en vergelijkt dit met de documentatie. Daarnaast worden verschillende database systemen met elkaar vergeleken. 

Drie verschillende database management systemen worden getest met behulp van de benchmarking tool, waarna de resultaten geanalyseerd worden. Op basis van deze resultaten komt verschillend gedrag tussen de systemen tot uiting zoals op een verschillende wijze omgaan met gelijktijdig lezen en schrijven van een data-element. 

De installatie en configuratie van de benchmarking tool en de reeds geteste database systemen is geautomatiseerd om te kunnen opstellen en uitvoeren zonder voorkennis van de specifieke systemen. Met een beperkte kennis en kost kan de tool ook uitgevoerd worden op andere database systemen. 


\end{abstract}

% Een lijst van figuren en tabellen is optioneel
%\listoffigures
%\listoftables
% Bij een beperkt aantal figuren en tabellen gebruik je liever het volgende:
%\listoffiguresandtables
% De lijst van symbolen is eveneens optioneel.
% Deze lijst moet wel manueel aangemaakt worden, bv. als volgt:
%\chapter{Lijst van afkortingen en symbolen}
\section*{Afkortingen}
\begin{flushleft}
  \renewcommand{\arraystretch}{1.1}
  \begin{tabularx}{\textwidth}{@{}p{12mm}X@{}}
    IMP   & Infrastructure Management Platform \\
    DBMS   & Databasemanagementsysteem \\
    RDBMS   & Relationeel Databasemanagementsysteem \\
    Range Query & Het opvragen van een set van records met behulp van een enkele query //
    BASE & //
    ACID & 
  \end{tabularx}
\end{flushleft}
\section*{Symbolen}
\begin{flushleft}
  \renewcommand{\arraystretch}{1.1}
  \begin{tabularx}{\textwidth}{@{}p{12mm}X@{}}
    42    & aaa \\
  \end{tabularx}
\end{flushleft}

% Nu begint de eigenlijke tekst
\mainmatter

\chapter{Inleiding}
De hedendaagse meest gebruikte database management systemen (DBMS's) zijn de relationele DBMS's of NoSQL systemen \cite{dbengine-ranking}. In deze thesis beschrijft een testmethode om beide systemen te kunnen vergelijken op basis van consistentie en beschikbaarheid, welke daarna toegepast wordt als voorbeeld op MongoDB, HBase en Pgpool-II (een uitbreiding van PostgreSQL). 

Het RDBM is er gekomen onder invloed van het artikel van E. Codd over het relationele model in 1969 \cite{codd1970relational}. Het sleutelconcept van het relationele model is dat de data georganiseerd is in tabellen, gekoppeld door sletuels (constraints). Dit concept leidt tot een vermindering van de redundante data. 
Voorbeelden van populaire relationele DBMS's (RDBMS's) zijn Oracle, MySQL en PostgreSQL. 

De NoSQL databases zijn een nieuwe generatie van systemen, de NoSQL beweging is gestart in 2000 en staat voor '\textit{Not only SQL}'. Deze systemen zijn er gekomen als op de globalisering van de computer systemen. Met een geografische spreiding van de verschillende datacentra konden de RDBMS niet om?. Dit leidde tot de nood voor meer flexibele databases, een lagere complexiteit, hogere doorvoer van data, horizontale schaalbaarheid en het draaien op commodity hardware. NoSQL DBMS proberen hieraan te voldoen met voorbeelden als Google BigTable, Amazon Dynamo, HBase, MongoDB, ... \cite{Strauch.NoSQL} 

In de volgende sectie zullen beide systemen in meer detail aanbod komen, waarna de huidige staat voor het kwantitatief vergelijken van de systemen aan bod. Tenslotte zullen de doelstellingen en de bijdragen van de thesis toegelicht worden. 

\section{Relationele en NoSQL DBMS's} 
Op dit moment zijn de meest gebruikte DBMS's de relationele en NoSQL systemen, maar wat dit net inhoudt en wat de verschillen tussen beiden zijn, zal in deze sectie in meer detail aanbod komen. Eerst zullen het relationele DBMS besproken worden, gevolgd door NoSQL een een korte bespreking van de grootste verschillen. 

\subsection{Relationele database}
Een RDBMS is een DBMS gebaseerd op relationele model voor het structuren van de database.

Het relationele model is vertrekt van theoretische wiskundige principes als set-theorie en eerste-orde predicaten logica. Het model organiseert de data in tabellen en relaties tussen de tabellen. De tabel heeft kolommen die verschillende velden voorstellen waarbij elke rij een collectie van gerelateerde datawaardes is. De relaties tussen de verschillende tabellen toont hoe deze bij elkaar horen. Een belangrijke eigenschap is dat de tabellen en relaties genormaliseerd worden, hiermee wordt redundante informatie verwijderd. Dit zorgt voor een hogere data integriteit en een vermindering in data anomalieën die kunnen optreden bij een update.\cite{Elmasri:2010:FDS:1855347} \\
De normalisatie kan geïllustreerd worden met het korte voorbeeld van figuur \ref{fig:Relationeel-Model-Normalisatie}: de professor voor een vak zal bij elke student hetzelfde zijn, het veranderen van een professor voor een vak zou in het eerste geval een update van alle ingeschreven studenten inhouden, in het tweede geval is dit maar de aanpassing van een enkel record, hetzelfde geldt voor de student. \\
Interactie met de RDBMS gebeurt op basis van SQL (Structured Query Language), een taal gebaseerd op de relationele logica. SQL geeft uitgebreide query mogelijkheden aan de gebruiker van de software.   
\begin{figure}[ht!]
\centering
\includegraphics[width=\linewidth]{img/Relationeel-Model-Normalisatie.png}
\caption[Relationeel datamodel (a) zonder en (b) met normalisatie]{Relationeel datamodel (a) zonder en (b) met normalisatie}
\label{fig:Relationeel-Model-Normalisatie}
\end{figure}

Een belangrijk concept in een relationele database is ACID, welk voor betrouwbare en robuuste transacties zorgt

\paragraph{Atomair (\underline{A}tomicity)} Een database transactie moet oftewel volledig uitgevoerd worden oftewel heeft geen enkele bewerking plaatsgevonden. 

\paragraph{Consistent (\underline{C}onsistency)} Een transactie behoudt consistentie als de volledige uitvoering van de transactie de database van één consistente staat naar een andere brengt. Een consistente staat is een staat die ervoor zorgt dat waardes van een instantie consistent zijn met de andere waarden in dezelfde staat. Een voorbeeld is het overschrijven van \euro{50} van persoon A naar B, op het einde moet de totale som nog steeds gelijk zijn, A \euro{50} minder en B \euro{50} meer. Een inconsistente staat zou zijn dat enkel A \euro{50} minder heeft, maar B nog steeds evenveel. 

\paragraph{Geïsoleerd (\underline{I}solation)} Een transactie moet uitgevoerd worden alsof ze volledig voor of na andere transacties heeft plaatsgevonden. 

\paragraph{Duurzaam (\underline{D}urability)} Een voltooide transactie kan later niet ongedaan gemaakt worden.

Deze verschillende concepten bieden de garanties welke de gebruiker kan gebruiken voor zijn systeem. Daartegen over staat wel dat dit de complexiteit van de RDBMS groeit, ook indien dit voor bepaalde toepassingen misschien niet nodig is.

\subsection{NoSQL database\cite{Strauch.NoSQL}}\label{sec:eventualconsistency}
NoSQL DBMS zijn ontstaan door een groei en globalisering van de computersystemen en de bijhorende databases. Een RDBMS is gebouwd met een 'one size fits all'-gedachte, maar deze systemen hiermee complexiteit die voor bepaalde toepassingen niet nodig is. NoSQL systemen bestaan in verschillende variëteiten, elk met hun eigen eigenschappen en toepassingsgebied om zo de complexiteit te verminderen. Tussen deze verschillen is er een rode draad te vinden vergeleken met een RDBMS:
\begin{itemize}
	\item \textbf{Lagere complexiteit}: NoSQL systemen bieden minder opties en garanties dan de RDBMS, bepaalde applicaties hebben enkel nood aan een deel van de garanties. Bijvoorbeeld in een sociale netwerk moet een post niet onmiddellijk beschikbaar zijn voor al de vrienden van een persoon, maar dit mag even duren.
	\item \textbf{Hogere doorvoer}: Talrijke NoSQL systemen bieden een hogere doorvoer van data aan wat in veel gevallen een gevolg is van de lagere complexiteit of door de hulp van andere bewerkingen zoals MapReduce \cite{dean2008mapreduce}.
	
	\item \textbf{Horizontale schaalbaarheid en werkend op commodity hardware}: Waar grote RDBMS's werken met dure high-end systemen, was het bedoeling van NoSQL databases ondersteuning te bieden aan een veelvoud van geclusterde eenvoudige machines (commodity hardware). \\
	Horizontale schaalbaarheid staat voor het toevoegen extra machines aan een systeem voor extra resources, in tegenstelling tot verticale schaalbaarheid waar een krachtiger machine wordt gebruikt voor de opschaling. De horizontale opschaling wordt tot uitvoering gebracht door de data van een enkele database of tabel te verspreiden over verschillende machines die elk maar voor een deel van de data verantwoordelijk zijn en moeten opslaan.\\
	NoSQL systemen combineren deze twee elementen en bieden hierdoor een schaalbaar systeem aan met basis componenten.
	\item \textbf{Datamodel dichter bij objecten}: De meeste NoSQL systemen zijn zodanig ontworpen dat deze de vertaling van objecten naar opslag eenvoudiger maken t.o.v. RDBMS's. RDBMS zijn ontworpen voor het ontstaan van object georiënteerde programmeertalen en heeft de nood aan de vertaling van een object naar de databasestructuur. Bij het ontwerp van NoSQL werd er hiermee onmiddellijk rekening gehouden.  
\end{itemize}  \noindent
Deze verschillende argumenten leiden vervolgens tot BASE, een tegenreactie op ACID. \noindent 
\begin{itemize}
 \item Basis beschikbaarheid (\textbf{B}asically \textbf{Availability}): het DBMS biedt lees- en schrijfacties aan bij het falen van één of meerdere falende instanties. De ondersteuning is afhankelijk van systeem tot systeem samen met de configuratie
 \item \textbf{S}oft State: De data moet op een bepaald moment niet volledig consistent zijn. 
 \item Eventuele consistentie (\textbf{E}ventual Consistency): De database zal na enige tijd in een consistente status uitkomen, het is mogelijk dat oudere data tijdelijk leesbaar is. Eventuele consistentie kan op zijn beurt opnieuw onderverdeeld worden in 4 categorieën \cite[slide 16]{lipcon2009design}:
 	\begin{itemize}
 		\item \textit{Read your own writes} consistentie: Ongeachte van de server waarop een gebruiker leest, zal hij zijn schrijfactie onmiddellijk correct lezen. 
 		\item \textit{Session} consistentie: De gebruiker zal zijn schrijfactie onmiddellijk kunnen lezen binnen dezelfde sessie, een sessie is hierdoor meestal gelimiteerd tot een enkele database server. 
 		\item \textit{Casual} consistentie: Als een gebruiker versie X leest en vervolgens versie Y schrijft, zal elke gebruiker die versie Y leest ook versie X lezen.
 		\item \textit{Monotonic Read} consistentie: Dit levert monotone tijdsgaranties dat een gebruiker enkel recentere data versies in de toekomst zal lezen. 
 	\end{itemize}
\end{itemize}
De BASE eigenschappen kunnen gekoppeld worden aan de CAP theorie van Erik Brewer\cite{Brewer:2000:TRD:343477.343502}. CAP zegt dat een gedistribueerd systeem maar twee van de 3 CAP elementen kan ondersteunen: consistentie, beschikbaarheid en partitie tolerantie. De beschikbaarheid betekent dat bij het falen van een instantie er nog steeds schrijfbewerkingen mogelijk zijn. Bij partitie tolerantie kan het systeem overweg met het opgesplitst zijn van instantie door een niet werkende netwerk verbinding. De definitie van consistentie is hier anders als bij ACID: bij CAP is er sprake van consistentie als het DBMS zich gedraagt alsof er maar een enkel kopie van de data is.

\subsubsection{Classificatie van NoSQL systemen}
Er zijn vele NoSQL systemen ontworpen gedurende de laatste jaren, elk met hun eigen variëteit, functionaliteit en populariteit. Er bestaan verschillende manieren om de systemen te classificeren, maar één van de meest gebruikte doet dit op basis de data modellering. Een korte vergelijking op basis van deze bevindt zich in tabel \ref{table:selectie-classificatie}.  

\begin{table}[!h]
	\resizebox{\textwidth}{!} {
		\begin{tabular}{l l l l l l}
			\textbf{Soort} & \textbf{Performantie} & \textbf{Schaalbaarheid} & 			\textbf{Flexibiliteit} & \textbf{Complexiteit} & \textbf{Functionaliteit} \\ \hline
			Column & hoog & hoog & gematigd & laag & minimaal \\
			Document & hoog & variabel(hoog) & hoog & laag & variabel (laag) \\
			Graph & variabel & variabel & hoog & hoog & graph theory \\
			Key-Value & hoog & hoog & hoog & geen & variabel (geen) \\
		\end{tabular}
	}
	\caption{Classificatie en categorisatie van NoSQL DBMS's door Scofield en Popescu. \cite{categorizatie-sco10} \cite{categorizatie-pop10b} }
	\label{table:selectie-classificatie}
\end{table} 

\paragraph{Column Model}In een column-gebaseerd systeem wordt de data opgeslagen per kolom in plaats van de traditionele manier, per rij. Deze aanpak werd in eerste instantie gedaan voor analyse van business intelligentie. Het systeem is geïnspireerd door de paper van Google’s Bigtable \cite{chang2008bigtable}. \cite{Strauch.NoSQL}

\paragraph{Graph Model} In een grafen model, wordt de data voorgesteld en opslagen volgens de grafen theorie: knopen, lijnen en eigenschappen op de knopen en lijnen. \cite{bollacker2008freebase}.   

\paragraph{Key-Value Model} Key-Value systemen hebben een heel eenvoudig data model, data kan opgeslagen, opgevraagd en verwijderd worden op basis van een key. De informatie die in de database zit, is de waarde voor die key. \\
Met dit eenvoudig model en functionaliteit die weinig complexiteit introduceren, kan er gestreefd worden naar een hoge performantie, schaalbaarheid en flexibiliteit. \cite{Strauch.NoSQL}

\paragraph{Document Model} Document systemen zijn volgens vele de volgende stap in key-value systemen, waar deze complexere structuren toe laten, dit door middel van meerdere key/value paren per element. \cite{Strauch.NoSQL} \\
Een document moet geen vaste structuur hebben maar elk document op zich kan verschillende velden hebben, dit kan bijvoorbeeld toegepast worden bij boeken. Waar een bepaald boek een recept is, kan een ander een deel zijn van een trilogie. Bij het eerste kan de kooktijd opgeslagen worden en bij de tweede een referentie naar de andere boeken. \cite{Strauch.NoSQL}


\subsection{Bespreking van verschillende DBMS's}
Databases uit 4 categorieën komen verder aanbod, er is gekozen om de Graph NoSQL DBMS's niet te bespreken. Graph NoSQL DBMS's zijn bedoeld voor de opslag van data van grafen. Deze is significant verschillend van de andere categorieën en hierdoor niet opgenomen. 
 
\begin{itemize}
\item Column NoSQL DBMS's: Cassandra, HBase
\item Document NoSQL DBMS's: Apache CoucheDB, MongoDB
\item Key-Value NoSQL DBMS's: LightCloud (Tokyo), MemCache, Redis, Riak, Project Voldemort
\item Relationele DBMS's: MySQL, Pgpool-II (PostgreSQL)
\end{itemize}

Deze keuze van deze systemen is gebaseerd op de paper van Christophe Strauch \cite{Strauch.NoSQL}. Een korte bespreking van de verschillende systemen kan gevonden worden in bijlage \ref{sec:BesprekingDBMS}.


\section{Vergelijking van DBMS's naar performantie en CAP}
Bij de ontwikkeling van verschillende systemen is er een keuze naar welk DBMS er kozen wordt. De systemen verschillen en hebben elk hun eigen toepassingsgebied. Zoals besproken hierboven kan een opsplitsing naar het datamodel gemaakt worden, of in meer detail naar de ondersteunde database bewerkingen. 

Maar de systemen kunnen ook verschillende performantie of een keuze in het CAP theorema gekozen hebben.  In dit gedeelte zal er gekeken worden welke methodes er al beschikbaar zijn voor het kwantitatief vergelijken van de performantie, consistentie en beschikbaarheid en mogelijke resultaten. 

\subsection{Performantie benchmarking}
Indien men verschillende DBMS's wilt vergelijken bestaan er al enkele tools en studies om de performantie te kunnen vergelijken. Een blogpost van A. Popescu \cite{PopescuBenchmarkOverview} geeft een overzicht van verschillende benchmarking tools. 

Als eerste hebben vele DBMS's \textbf{interne benchmarking tools}, waarmee de database op verschillende configuraties kan getest en vergeleken worden. Deze resultaten zijn nuttig na de keuze van het DBMS. Het systeem kan getest worden bij het variëren van de parameters en het uitzoeken wat de bottleneck is in een bepaald systeem. Een voorbeeld hiervan is mongoperf\footnote{\url{http://docs.mongodb.org/manual/reference/program/mongoperf/}} voor MongoDB. 

Andere studies focussen op het testen van verschillende systemen en daarbij kunnen verschillende doelstellingen zijn: het ontwikkelen van een breed toepasbare tool, het testen van een grote verscheidenheid van DBMS's of het testen van een specifieke categorie van systemen. Elke van deze benchmarking brengt nieuwe kennis van de systemen maar heeft ook zijn beperkingen. Het totaal pakket van al de testen kan een gebruiker de informatie geven om een beter gefundeerde keuze te maken. 

Een eerste categorie van deze externe tools is het \textbf{ontwikkelen van een tool} voor verschillende systemen. Dit heeft als grote voordeel dat andere gebruikers nadien de testen opnieuw kunnen uitvoeren met de systemen in hun configuratie. Het is namelijk niet gegarandeerd dat het resultaat van een jaar geleden gelijkaard is met de nieuwste versie.
Het grootste nadeel is de testen die kunnen uitgevoerd worden, er is een grote variëteit aan systemen elk met hun eigen datastructuur en query mogelijkheden. De tool moet dus een gemeenschappelijke subset zoeken en enkel dit soort queries kunnen getest worden. Een voorbeeld van een dergelijke tool is YCSB\cite{cooper2010benchmarking}. Deze tool kan elk DBMS's testen zolang een basisset van 5 queries ondersteund wordt: het invoegen, updaten, verwijderen en opvragen van een enkel record met daarnaast ook de mogelijkheid tot scan queries, met behulp van 1 query een verzameling van records tegelijk op te vragen. \\
Sommige systemen ondersteunen bepaalde queries niet rechtstreeks maar bevatten wel de functionaliteit om deze met behulp van meerdere achtereenvolgende bewerkingen te implementeren. Bijvoorbeeld een update kan geïmplementeerd worden door het opvragen, verwijderen en vervolgen invoegen van het aangepaste record. 

Een volgende categorie zijn de \textbf{resultaten van gerelateerde DBMS's}, dit zijn voornamelijk systemen met hetzelfde datamodel. Het grote voordeel hieraan is dat deze systemen in de meeste gevallen een vrij gelijkaardige set aan query mogelijkheden bevatten waardoor er meer diepgang is dan tussen meer verschillende systemen. Een voorbeeld van zulk onderzoek is gedaan door P. Pirzadeh et al\cite{pirzadeh2011performance} voor de key-value systemen, meer specifiek is er gefocust op het uitvoeren van range queries tussen Cassandra, HBase en Voldemort.  \\
In deze categorie vallen ook de resultaten die meestal getoond worden op de website van de DBMS's, een vergelijkende benchmark met andere soortgelijke systemen. Hoewel de resultaten niet altijd volledig objectief zijn, kan de gevolgde test methode wel interessant zijn. Een voorbeeld van deze studie is de Key-Value benchmarking van VoltDB\cite{huggkey} waar Cassandra en VoltDB vergeleken worden, een belangrijke kanttekening is dat de auteur zelf al aanhaalt dat de systemen vrij verschillend zijn.

Als laatste categorie, zijn er de \textbf{resultaten van verschillende DBMS's} waar verschillende soorten systemen met elkaar getest worden. De belangrijkste voordeel is dat er resultaten zijn die verschillende soorten met elkaar vergelijken en waardoor niet alleen verschillen in het datamodel kunnen vergeleken worden in toekomstige studies maar ook performantie verschillen. Het nadeel is dat er een gemeenschappelijke subset gevonden moet worden, hierdoor kunnen bepaalde databases hun kracht net niet laten zien. Enkele van deze onderzoeken zijn \cite{tudorica2011comparison} en \cite{rabl2012solving}. Deze laatste maakt gebruik van de YCSB tool die hierboven besproken was. 


\subsection{Consistentietesten}
Bij een gedistribueerd systeem kunnen er verschillende keuzes gemaakt worden naar synchrone of asynchrone replicatie en welk soort consistentie er aangeboden wordt. In de documentatie van DBMS's worden er beloftes gemaakt, maar hoe is de consistentie in de realiteit?

Een recent artikel \cite{golab2014eventually} (maart 2014), stelt dat er momenteel nauwelijks gekwantificeerde methodes bestaan om de eventuele consistente te meten. In hun artikel stellen zij twee mogelijke methoden voor: de actieve of passieve analyse. \\
De \textbf{actieve} analyse bestaat uit het wegschrijven van data waarna men hoe lang het duurt vooraleer alle servers de nieuwe waarde hebben. 
Bij de \textbf{passieve} analyse kijkt men langs de gebruikerskant. Leest de gebruiker altijd de laatste waarde (=strikte consistentie)? Is het mogelijk dat een nieuwe waarde al wordt gelezen voor de schrijfactie voltooid is? \\
Beide analyses hebben hun eigenschappen, de actieve analyse is gericht op het database systeem en zijn server. 
Bij de passieve analyse is georiënteerd naar de gebruiker toe, hoe moet deze zijn toepassingen aanpassen, wat zijn de garanties die geleverd worden aan de gebruiker? 

Voornamelijk naar actieve analyse is er al kwantitatief onderzoek verricht. Onder andere Duitse onderzoekers hebben op het Amazon S3 platform getest hoe lang het duurt vooraleer data geschreven in MiniStorage beschikbaar is voor alle gebruikers op al de verschillende servers. \cite{bermbach2011eventual}. \\
Daarnaast zijn er ook 2 interessante resultaten gevonden: allereerst heeft het Amazon S3 systeem geen monotone lees consistentie, daarnaast bleek het inconsistentie interval voor een bepaald record periodiek verloop te hebben dat niet door de onderzoekers verklaard konden worden. 

De YCSB software van hierboven is door onderzoekers in de VS uitgebreid naar YCSB++\cite{patil2011ycsb++} waardoor deze meer ondersteuning heeft voor het meten van systeembelasting maar ook voor de consistentie-eigenschappen. Enkele geteste systemen zijn in principe strikt consistent, zoals HBase, maar deze worden eventueel consistent door het gebruiken van buffers bij de gebruiker. Vervolgens testen zij hoe lang het duurt voor de data ook gelezen kan worden. De vertraging is sterk afhankelijk is van het aantal acties van de schrijvende gebruiker: indien er meer geschreven wordt, zal de buffer sneller verzonden worden naar de server en dus sneller beschikbaar zijn voor andere gebruikers. \\
Hoewel zij stellen dat er ook testen zijn gedaan naar eventuele consistentie voor Cassandra en MongoDB, zijn de resultaten niet beschikbaar in het artikel of op de website. 

Andere onderzoeker\cite{wada2011data} doen analyse op Amazon SimpleDB. In de situatie wordt getest of de database read-your-own-writes en monotone consistentie ondersteund. Aan het eerste is niet voldaan met \textit{eventual consistency read}. Met minder dan 500 ms tussen het einde van de schrijfactie en het begin van de leesactie, wordt er slechts 33\% van de nieuwe data gelezen, zodra er meer als 500 ms gewacht wordt, gaat dit naar 99\%. Ook is er geen monotone consistentie omdat er van verschillende servers kan gelezen worden bij opvolgende leesacties, sommige zullen de data al wel hebben, anderen niet. 

Bij Netflix heeft men aan passieve analyse gedaan op hun Cassandra systeem \cite{kalantzisnetflix} waar zij in hun testen geen consistentieproblemen vonden naar de gebruiker toe. Er is geen vermelding hoeveel vertraging er zit tussen beide transacties. Volgens hun gaat het meer om de perceptie dat data verkeerd kan gelezen worden en de angst van het middle management. 

\subsection{Beschikbaarheidstesten}
Een derde verschilpunt is hoe de systemen omgaan met het falen van een enkele server en dit onder verschillende opties: Het is mogelijk dat deze tijdelijk uitgeschakeld wordt wegens onderhoud. Het kan gaan om een onverwachte crash van de software op een server of een crash van een volledige server, tenslotte kunnen er ook nog netwerkproblemen optreden waardoor deze (tijdelijk) niet beschikbaar is. 

Nu hoe gaan deze systemen om het falen en terug online brengen van de systemen? Zijn er geen acties mogelijk op de server, worden de connecties tijdelijk verbroken, is er een verhoogde of verlaagde vertraging op de transacties? Detecteert het systeem automatisch wanneer de oorspronkelijke server terug online komt of moet gemeld worden om de server terug te gebruiken? In een NoSQL DBMS waar gewerkt wordt commodity hardware, zal het falen regelmatig gebeuren en verschillende systemen reageren anders op deze acties. 

Uit de literatuurstudie zijn er geen vergelijkende studies voor database systemen gevonden. Er is voor de meeste DBMS's informatie te vinden op de website hoe zij in een gedistribueerde omgeving werken zoals het al dan niet gebruik van sessies en de duur van een sessie. Maar voor het effect in de praktijk, is het handmatig testen. 

\section{Doelstelling en bijdrage}\todo{Loop hier nog is over}
In de literatuurstudie komt naar voor dat er een gebrek is aan kwantitatieve methodes om DBMS's te vergelijken naar consistentie en beschikbaarheid. 

Het eerste doel van deze thesis is om benchmark te ontwikkelen die het gedrag in de praktijk kan testen. Deze testen zullen het uitvallen van een service, server en netwerk simuleren voor de beschikbaarheidstesten. Bij de consistentietesten wordt er een tool aangeleverd die verschillende types van eventuele consistentie kan testen. 

Daarnaast zal aangetoond worden dat de testmethode werkt door het toepassen op drie voorbeeldsystemen: HBase, MongoDB en Pgpool-II(uitbreiding van PostgreSQL). Door het kiezen van verschillende systemen met een verschillend datamodel, kan de testmethode een kwantitatieve vergelijking ten opzichte van elkaar mogelijk maken. 

Tenslotte is het doel om de testen eenvoudig te kunnen uitvoeren voor andere gebruikers, op deze manier kan de data gecontroleerd worden. Daarnaast wordt er ook de mogelijkheid aangeboden om nieuwere versies en andere infrastructuren te kunnen testen.

Met deze drie doelstellingen wordt er een testmethode, een werkende benchmarking tool en resultaten aan een gebied waar er nog nauwelijks kwantitatieve methodes zijn. 

\section{Conclusie}
Deze thesis zal handelen over het ontwikkelen en uitvoeren van een benchmark systeem voor consistentie en beschikbaarheid naar relationele en NoSQL databases. Momenteel zijn er weinig kwantitatieve methodes en resultaten naar beschikbaarheid en consistentie.  

In het vervolg van deze tekst, zal in hoofdstuk \ref{sec:methodiekvantesten} een algemene testmethode voor consistentie en beschikbaarheid voorgesteld worden. In hoofdstuk \ref{sec:implementatie} wordt besproken hoe de voorgestelde methode in de praktijk geïmplementeerd wordt. Daarna worden de observaties voorgesteld in hoofdstuk \ref{sec:observaties} met een analyse en verklaring van de observaties in hoofdstuk \ref{sec:analyse}. Een conclusie volgt tenslotte in hoofdstuk \ref{sec:conclusie}. 
\chapter{Methodiek van de testen}	



\section{Vereisten van de systemen}

\begin{itemize}
\item \textbf{Persistentie}: 
\item \textbf{Data Distributie}
\item \textbf{Replicatie}
\item \textbf{Open-source}
\end{itemize}

\subsection{Gekozen systemen}
\subsubsection{HBase}
\subsubsection{MongoDB}
\subsubsection{Pgpool-II (PostgreSQL)}
\chapter{Implementatie}\label{sec:implementatie}
In het vorige hoofdstuk is uitgelegd wat de methode is voor het testen, maar hoe vertalen deze vereisten naar een werkende test programma? In dit hoofdstuk zal deze vertaling uitgelegd worden die op deze thesis is toegepast. 

In het eerste deel wordt uitgelegd wat de selectie criteria waren om HBase, MongoDB en Pgpool-II (PostgreSQL) te verkiezen. Daarna zullen drie systemen in meer detail uitgelegd worden om zo de specifieke architectuur uit de doeken te doen. Daarna wordt de testsoftware besproken met de testconfiguratie. \todo{meer?} 
 
\section{Selectie van de DBMS's}
Voor de selectie van de systemen is er onderzocht of een systeem een bepaalde eigenschap al dan niet ondersteunt. In het totaal zijn er 5 verschillende eigenschappen waarop de selectie is gebaseerd. 

\paragraph{Vrije software} Om testen tussen verschillende DBMS's te kunnen vergelijken op een gelijkaardige infrastructuur, is het nodig dat deze software kan geïnstalleerd worden op de eigen infrastructuur. In deze thesis is er gefocust op systemen die gratis aangeboden worden. 

\paragraph{Persistentie} Voor het testen van de beschikbaarheid van de data, is het een voordeel dat de data op harde schijf aanwezig is: bij een herstel dient er minder data over het netwerk gestuurd te worden. Om deze reden hebben persistente systemen een voorkeur op deze die de data enkel in geheugen houden. 

\paragraph{Replicatie} Eén van de testen is de beschikbaarheidstest. Indien de data maar op een enkele server opgeslagen is, zal de data op de uitgeschakelde server niet langer beschikbaar zijn. Met replicatie zal de data op verschillende servers opgeslagen worden en kan de data in theorie nog beschikbaar zijn in het geval van een enkele uitgeschakelde server. 

\paragraph{Data distributie} Het is de bedoeling om systemen te testen die een grote hoeveelheid data kunnen opslaan. Om aan deze vereiste te voldoen, is het nodig dat elke server niet al de data opslaat bij een grote dataset. 

\paragraph{Ondersteuning voor verschillende query methodes} Bij de testen worden er 5 soorten queries uitgevoerd: invoegen, aanpassen, verwijderen en het opvragen van een individueel of meerdere record. De DBMS moet ondersteuning voor deze queries. De eerste 4 kunnen in al de systemen geïmplementeerd worden met één of meerdere queries. Maar het opvragen van meerdere queries, een scan query, is in bepaalde systemen niet ondersteund. Deze scan query is een query waar het record met een bepaalde sleutel wordt opgevraagd en een aantal records dat hierop volgt, het is \textit{geen} query met een begin en eind sleutel. 

Voor alle systemen van het boek \cite{Strauch.NoSQL},besproken in sectie \ref{sec:BesprekingDBMS}, is het eerste criterium voldaan. Een vergelijking voor de vorige  zijn samengevat in tabel \ref{table:vergelijkingNosql}.

Bij de selectie is er naast de 4 criteria, ook gekozen voor systemen van verschillende datamodellen. Samen met mijn collega Arnaud Schoonjans \cite{thesisArnaud}, zijn er in 7 verschillende systemen verder onderzocht en als modelsysteem gekozen. Voor deze thesis zijn dit HBase, MongoDB en Pgpool-II verder onderzocht, in de thesis van mijn collega zijn dit Cassandra, Apache CouchDB, Riak en MySQL. \todo{Extra uitleg waaro?}
 
% Table generated by Excel2LaTeX from sheet 'Sheet1'
\begin{table}[htbp]
  \centering
  \resizebox{\columnwidth}{!}{%
    \begin{tabular}{ll|lllll}
          &       & \multirow{2}[1]{*}{Persistentie} & \multirow{2}[1]{*}{Replicatie} & \multirow{2}[1]{*}{Datadistributie} & \multicolumn{2}{c}{Query soort} \\
    
          &       &       &       &       & Aanpassen & Scan \\ \hline
    \multirow{2}[1]{*}{Column} & Cassandra & Ja    & Master-Master & Ja    & Ja    & Half \\
          & HBase & Ja    & Master-Slave & Ja    & Ja    & Ja \\
          
          \hline
    \multirow{3}[0]{*}{Document} & Apache  & \multicolumn{1}{l}{\multirow{2}[0]{*}{Ja}} & \multicolumn{1}{l}{\multirow{2}[0]{*}{Master-Master}} & \multicolumn{1}{l}{\multirow{2}[0]{*}{Ja}} & \multicolumn{1}{l}{\multirow{2}[0]{*}{Nee}} & \multicolumn{1}{l}{\multirow{2}[0]{*}{Ja}} \\
          & CoucheDB & \multicolumn{1}{l}{} & \multicolumn{1}{l}{} & \multicolumn{1}{l}{} & \multicolumn{1}{l}{} & \multicolumn{1}{l}{} \\
          & MongoDB & Ja    & Master-Slave & Ja    & Ja    & Ja \\
          
          \hline
    \multirow{6}[0]{*}{Key-Value} & LightCloud & \multicolumn{1}{l}{\multirow{2}[0]{*}{Ja}} & \multicolumn{1}{l}{\multirow{2}[0]{*}{Master-Master}} & \multicolumn{1}{l}{\multirow{2}[0]{*}{Ja}} & \multicolumn{1}{l}{\multirow{2}[0]{*}{Nee}} & \multicolumn{1}{l}{\multirow{2}[0]{*}{Ja}} \\
          & (Tokyo) & \multicolumn{1}{l}{} & \multicolumn{1}{l}{} & \multicolumn{1}{l}{} & \multicolumn{1}{l}{} & \multicolumn{1}{l}{} \\
          & MemcacheDB & Ja    & Master-Slave    & Nee   & Nee   & Ja \\
          & Redis & Half & Master-Slave & Nee  & Ja    & Half \\
          & Riak  & Ja    & Master-Master & Ja    & Nee   & Half \\
          & Voldemort & Ja    & Master-Master & Ja    & Nee   & Nee \\
          
          \hline
    \multirow{3}[0]{*}{Relationeel} & MySQL & Ja    & Master-Slave & Nee   & Ja    & Ja \\
    	  & PostgreSQL & Ja    & Master-Slave & Nee   & Ja    & Ja \\
          & Pgpool-II & \multicolumn{1}{l}{\multirow{2}[0]{*}{Ja}} & \multicolumn{1}{l}{\multirow{2}[0]{*}{Master-Slave}} & \multicolumn{1}{l}{\multirow{2}[0]{*}{Ja}} & \multicolumn{1}{l}{\multirow{2}[0]{*}{Ja}} & \multicolumn{1}{l}{\multirow{2}[0]{*}{Ja}} \\
          & (PostgreSQL) & \multicolumn{1}{l}{} & \multicolumn{1}{l}{} & \multicolumn{1}{l}{} & \multicolumn{1}{l}{} & \multicolumn{1}{l}{} \\
    \end{tabular}%
    }
    \caption{Ondersteuning van de besproken DBMS's naar de selectie criteria. \newline
    Bij \textit{Redis} is er sprake van een snapshot of een log voor de persistentie, de eigenlijke database wordt enkel in het geheugen gehouden. Hierdoor is er maar half sprake , hierdoor kan de database herstelt worden maar is deze niet in het geheugen.\newline 
    Bij \textit{replicatie} zijn er 2 mogelijke configuraties: master-slave waarbij er verschillende instanties verschillende functies hebben en één de baas is, of master-master waarbij ze allemaal gelijk zijn. \newline 
    Bij \textit{aanpassen} zijn er systemen die voor een update al de verschillende kolom waarden nodig hebben of maar 1 kolom per waarde ondersteunen.\newline 
    Bij \textit{scan} is er bij enkele systemen enkel ondersteuning voor het lezen tussen 2 verschillende sleutels. Met het iteratief opvragen van elementen tussen 2 sleutels en het lezen van een beperkte hoeveelheid data, is het mogelijk om een scan query uit te voeren, maar dit is maar halve ondersteuning. }
  \label{table:vergelijkingNosql}%
\end{table}%

\section{Gedetailleerde bespreking van de model DBMS's}
In dit gedeelte zal elk model systemen in meer detail uitgelegd worden. Een gemeenschappelijk element bij al deze systemen is dat niet alle instanties dezelfde functie hebben, in andere DBMS's hebben alle instanties dezelfde functie bij het wat de installatie en configuratie kan vereenvoudigen. 

Voor elk van de geselecteerde systemen zal de aangeboden API besproken worden met een blik op de datastructuur, daarna zal de systeem architectuur besproken worden. 

\subsection{HBase}

\subsubsection{Data structuur\cite{george2011hbase}}
De data in HBase is gestructureerd in tabellen, bij het aanmaken wordt er een schema voor de tabel gemaakt. Voor elke tabel kunnen de verschillende kolommen meegegeven worden samen met een \textit{kolom familie} voor elke kolom, maar de kolommen kunnen ook gespecificeerd worden bij het schrijven van data. De gegevens per \textit{kolom familie} hebben dezelfde prefix en zullen fysisch samen opgeslagen worden. Indien verschillende kolommen tegelijk worden gelezen of geschreven, is het aangeraden om deze dezelfde \textit{kolom familie} te geven. 

De operaties beschikbaar in dit systeem zijn: get (verkrijgen), put (invoegen), scan en delete (verwijderen). Het aanpassen van gegevens wordt uitgevoerd via een put waarbij een enkele kolom waarde van een record kan aanpast worden. Een scan operatie heeft geen optie om het aantal op te halen records te bepalen. Maar HBase ondersteunt de optie om de batch grootte (bytes) te configureren. Doordat er geweten is hoe groot een individueel record is én hoeveel records er opgevraagd worden, kan de cache grootte zo bepaald worden. Op deze manier is er maar een enkele communicatie met de database nodig is. 

\subsubsection{Architectuur\cite{george2011hbase}}
De gedistribueerde versie van HBase is afhankelijk van 2 andere software systemen: Zookeeper\cite{hunt2010zookeeper} en Hadoop\cite{borthakur2007hadoop}. Hiermee volgt HBase de structuur van Google's BigTable\cite{chang2008bigtable} die op zijn beurt afhankelijk is van Chubby\cite{burrows2006chubby} en Google File System\cite{ghemawat2003google}. Een overzicht van de architectuur bevindt zich in figuur \ref{fig:Hbase-structure}. De 3 systemen van HBase zullen kort besproken worden, van HBase naar Zookeeper en Hadoop. 

\begin{figure}[ht!]
\centering
\includegraphics[width=\linewidth]{img/Hbase-structure.png}
\caption{Volledige systeemarchitectuur van HBase met Hadoop en Zookeeper. Bron \cite{ChinHBaseComprehensive}}
\label{fig:Hbase-structure}
\end{figure}

\paragraph{HBase\cite{george2011hbase}} HBase is een master/slave systeem welke bestaat uit een HMaster en een HRegionServer. De \textit{HMaster} is verbonden met Zookeeper en houdt op deze manier de status en verantwoordelijkheden van de HRegionServers in het oog. Daarnaast is deze ook verantwoordelijk voor het toewijzen van data verantwoordelijkheden. Onder andere wordt de data uit een tabel opgesplitst over verschillende regio's indien een tabel groeit, ook het toewijzen van een HRegion aan een HRegionServer.\\
De andere soort, een \textit{HRegionServer}, is verantwoordelijke voor de data van een regio en voor het beheren van deze regio's. Een regio is een deel van een tabel met daar in de feitelijke data die opgeslagen is in verschillende Hadoop datanodes. Een HRegionServer zal consistentie en atomaire queries afdwingen in HBase op een enkele record.  

\paragraph{Hadoop\cite{borthakur2007hadoop}} HBase maakt gebruik van het Hadoop Distributed File System (HDFS), een gedistribueerd file systeem ontworpen om te werken op commodity hardware met een hoge fout tolerantie. HDFS heeft een master/slave architectuur en bestaat uit een enkele \textit{namenode}, de master server, die de naamruimte en toegangscontrole onderhoudt, en \textit{datanode}s. De data wordt opgedeeld in blokken die door een verzameling van datanodes wordt opgeslagen. Aangezien niet elke node in deze verzameling zit, is er op deze manier data distributie. Deze master/slave configuratie zijn verschillende soorten van services die de administrator afzonderlijk moet opzetten. \\
In de deze configuratie van HBase, is HDFS de methode om data persistent op te slaan met automatische replicatie en data distributie. Er is ook ondersteuning om de opslag naar Amazon S3 te doen in een gedistribueerde omgeving of deze op de lokale harde schrijf op te slaan bij een configuratie met slechts 1 server.\cite{george2011hbase}

\paragraph{Zookeeper\cite{hunt2010zookeeper}} Zookeeper is een service voor het coördineren van gedistribueerde applicatie processen. Deze service biedt primitieven aan om synchronisatie, configuratieonderhoud en benaming te doen. Zookeeper is een gedistribueerd master/slave systeem dat ontworpen is om snel te zijn bij dominantie van leesoperaties.  \\
HBase gebruikt Zookeeper voor het bijhouden van de status van HRegionserver, hun locatie en hun verantwoordelijkheden. De sessie wordt toegekend die een HRegionServer bijvoorbeeld de verantwoordelijkheid voor een Region geeft voor de volgende minuut. Tijdens deze periode kan geen enkele andere HRegionServer een bewerking doen op deze Region, uitgezonderd met de toestemming van de verantwoordelijke server. \cite{george2011hbase} De duur van een sessie kan geconfigureerd worden in Zookeeper maar wordt in dit geval op de standaard 180 seconden gelaten. 

Dit is de globale structuur van het HBase systeem, in het totaal zijn er 5 verschillende soorten services: 2 voor Hadoop, 1 voor Zookeeper en 2 bij HBase. Enkele van deze services worden best gegroepeerd op een enkele instantie: de HDFS namenode, een Zookeeper instantie en de HMaster worden samen op een enkele instantie geplaatst. Hetzelfde geldt voor een datanode en een HRegionServer. Zeker deze laatste heeft een extra performantie invloed: HBase detecteert dat er lokale opslag van de data is en de regio zal steeds deze lokale opslag hebben. Dit zorgt bij leesacties voor een performantie verbetering aangezien de data lokaal gelezen kan worden.

De configuratie van de verschillende systemen gebeurt door middel van configuratiebestanden voor elke service waarna de verschillende systemen zich bij elkaar aanmelden en de volledige configuratie van Region's door het systeem zelf wordt gedaan.  

\subsection{MongoDB\cite{mongodb-manual}}

\subsubsection{Datastructuur}
De data in MongoDB is opgeslagen in een database, die op zijn beurt een collectie bevat. Het is niet nodig om een een database en collectie op voorhand aan te maken. Beiden worden automatisch aangemaakt bij het wegschrijven van data, indien de collectie nog niet bestaat. Een record is in MongoDB een document en elk record kan verschillende velden hebben. Er zijn uitgebreide query mogelijkheden om data in te voegen, aan te passen, te verwijderen of een scan uit te voeren. Er is ook ondersteuning voor MapReduce\cite{dean2008mapreduce}. 

Bij het schrijven van data, kunnen verschillende eisen gesteld worden voor het voltooien van de actie, startende met de actie is over het netwerk verstuurd, de primary heeft de data geschreven tot een meerderheid van de secondaries heeft de data weg geschreven. \\ Bij het lezen kan men kiezen om de data te lezen van de primary, secondary of de dichtstbijzijnde node. Afhankelijk van de gekozen acties, kan er verondersteld worden dat er een verschillende consistentie garantie zal zijn. Een overzicht van al de mogelijkheden, kan teruggevonden worden in tabel \ref{table:mongodb-query-opties}. Indien er in de tekst verder niet gespecificeerd wordt welke lees- of schrijfconfiguratie er wordt gebruikt, zijn dit de standaard methodes, respectievelijk primary en normal. 

\begin{table}[ht!]
	\centering
	\begin{tabular}{l|l}
		\multicolumn{2}{c}{\textbf{Leesconfiguratie}} \\ 
		\textbf{Benaming} & \textbf{Omschrijving} \\ \hline
		Primary & Enkel lezen van de primary \\
		\multirow{2}[1]{*}{PrimaryPreferred} & Lezen van de primary, behalve als de primary  \\
		& onbeschikbaar is, lees dan van secondary. \\
		Secundary & Enkel lezen van een secondary \\
		\multirow{2}[1]{*}{SecundaryPreferred} & Lezen van een secondary, behalve als er geen secondary  \\
				& onbeschikbaar is, lees dan van de primary. \\
		\multirow{2}[1]{*}{Nearest} & Lees van de instantie met de laagste netwerk \\
				& vertraging, ongeachte het een primary of secondary is. \\
		\multicolumn{2}{c}{\textbf{}} \\ 		
		
		\multicolumn{2}{c}{\textbf{Schrijf configuratie}} \\ 
		\textbf{Benaming} & \textbf{Omschrijving} \\ \hline
		Normal & Wacht tot weggeschreven naar het netwerk socket. \\
		\multirow{1}[1]{*}{Safe} & Wacht op bevestiging van de primary    \\
		
		\multirow{2}[1]{*}{fsync$\_$safe} & Wacht op bevestiging van de primary tot  \\
				& de data is weggeschreven naar harde schijf.  \\
		\multirow{2}[1]{*}{Replica acknowledged} & Wacht op bevestiging van primary  \\
				& en één secondary. \\
		\multirow{1}[1]{*}{Majority} & Wacht op bevesting van meerderheid van de servers  \\
	\end{tabular}
	\caption{MongoDB: Mogelijke configuraties bij lees- en schrijfbewerkingen}
	\label{table:mongodb-query-opties}
\end{table}

\subsubsection{Architectuur}
MongoDB is een DBMS dat de vereisten van replicatie en data distributie op een gelaagde manier tot uitvoering brengt. Eerst zullen instanties gecombineerd worden voor de replicatievereisten invullen, daarna zal horizontale schaalbaarheid ondersteund worden. 

\begin{figure}[ht!] 
\centering
	\subfigure[Drie leden van een replica set met een primary en 2 secondaries. ]{\label{fig:mongodb-replicaset} \includegraphics[width=0.40\textwidth]{img/mongodb-replica-set-primary-with-two-secondaries}}
	\hfill
	\subfigure[Een voorbeeld cluster voor productie met 2 toegangservers (mongos), 2 shards en 3 configuratieservers.]{\label{fig:mongodb-sharding} \includegraphics[width=0.50\textwidth]{img/mongo-sharded-cluster-production-architecture}}
	\caption{MongoDB Architectuur voor replicatie en datadistributie. Bron figuur link: \cite{mongodb-replicaset}, rechts: \cite{mongodb-shard}}
	\label{fig:mongodb-architectuur}
\end{figure}

\paragraph{Replicatie\cite{mongodb-replicaset}} Replicatie gebeurt door middel van een master/slave configuratie tussen verschillende \textit{MongoD} instanties, of in hun termen primary/secondary. Een verzameling van deze MongoD instanties wordt een \textit{replicaset} genoemd. Een replicaset verkiest zelf de primary die verantwoordelijk is voor het afhandelen van de schrijfacties. De data zal vervolgens gerepliceerd worden naar de secondaries. Of deze actie synchroon of asynchroon is, hangt af van de gekozen schrijfconfiguratie.  Het is slechts mogelijk om een instantie tot een enkele set toe te voegen. De data is beschikbaar zo lang er meer dan de helft van de servers beschikbaar zijn. 

\paragraph{Data distributie\cite{mongodb-shard}} Horizontale schaalbaarheid wordt in MongoDB bereikt door verschillende replicaset's of zelfstandige MongoD instanties te combineren tot een cluster. In het geval van een zelfdstandige instantie, zal de data niet gerepliceerd worden en wordt om deze reden niet aangeraden voor productie. Voor datadistributie bestaan er 3 verschillende type servers, sharding-, configuratie- en toegangsservers. Een overzicht is gegeven in figuur \ref{fig:mongodb-sharding}. 

\subparagraph{Shards} De data wordt verdeeld over de verschillende shards nadat is aangegeven dat men deze wilt verdelen over de cluster. Deze verdeling wordt automatisch aangepast indien een enkele shard te groot wordt. 
\subparagraph{Configuratie servers} De configuratie servers slaan de meta data van de cluster op zoals de verschillende shards en replicaset's. Deze configuratie set bestaat uit 1 tot 3 servers, voor productie zijn 3 servers aangeraden. Deze servers verdelen de data over de verschillende shards en zullen een de data herstructureren als deze te groot wordt. 
\subparagraph{Toegangsserver} De toegangsserver biedt toegang aan tot de cluster en vraagt de configuratie op aan de configuratie servers. Er kunnen een onbepaald aantal toegangsservers zijn in cluster. 

De configuratie van de verschillende delen bestaat uit verschillende technieken. Bij replicatie krijgt elke set een naam die in de configuratiebestanden van elke configuratie wordt gezet. Nadien wordt één instantie op de hoogte gebracht van de locatie van de andere instanties. Bij de cluster worden bij het opstarten van de toegangsservers de set van configuratieservers meegegeven, het opzetten van de verschillende shards gebeurt via een toegangsserver m.b.v. de API. 

\subsection{Pgpool-II (PostgreSQL)\cite{pgpool-doc}}
Pgpool-II kan op 4 verschillende manieren werken, in deze testen is er gekozen voor de replicatie optie omdat deze zowel replicatie, failover en online recovery aanbiedt en de leesbewerkingen verspreid. Er is de mogelijkheid om ook data distributie aan te bieden maar dit is niet getest. Door de datadistributie bovenop de replicatie te zetten, volgt MongoDB hetzelfde principe als MongoDB. 

Datastructuur en de architectuur van Pgpool-II in replicatie mode wordt nu in meer detail besproken. 

\subsubsection{Datastructuur}
De data structuur en query mogelijkheden van Pgpool-II zijn gelijklopend aan deze van PostgreSQL. Net zoals in PostgreSQL bestaat het systeem uit een schema die verschillende databases kan bevatten. Een database bestaat uit een verzameling van tabellen, een tabel bevat de records. Voor het opslaan van de data dient de volledige tabel met al de kolommen gespecificeerd zijn. 

Pgpool-II ondersteunt de volledige query mogelijkheden die in de testen nodig zijn. Er zijn enkele restricties ten opzichte van PostgreSQL die beschreven zijn op in de sectie \textit{Restrictions} van de documentatie\cite{pgpool-doc}. 

\subsubsection{Architectuur}
Een Pgpool-II infrastructuur bestaat uit 2 delen, een data en routing niveau, een overzicht is gegeven in figuur \ref{fig:Pgpool-structure}. 

\begin{figure}[ht!]
\centering
\includegraphics[width=0.5\linewidth]{img/Pgpool-structuur}
\caption{Systeemarchitectuur van Pgpool-II.}
\label{fig:Pgpool-structure}
\end{figure}

Het data niveau bestaat uit een individuele service uit een PostgreSQL installatie met extra functies, bestanden en een aanpassing aan enkele configuratie bestanden. Daarnaast moet er voor de online recovery ook ssh toegang voorzien worden tussen al de servers. De verschillende data machines hebben een master/slave structuur waar al de schrijfacties naar de master worden gestuurd en de leesoperaties zijn verdeeld over al de machines. De master doet aan synchronisatie met behulp van de \textit{Write-ahead-log} van PostgreSQL. Dit is een log waar al de verschillende schrijfacties worden opgeslagen. Door een andere server deze te laten uitvoeren is er een gelijke database. 

Op routing niveau draait een Pgpool-II service die als management service dient, hij bepaalt wie master en slave is, volgt de status op van de data services en doet aan online recovery. Op het moment dat een databaseverbinding wordt aangemaakt, kiest Pgpool-II naar welke PostgreSQL server dit gaat. Op deze manier wordt de leesbelasting verdeeld. 

Pgpool-II kan ook in de parallel mode werken zodat er de mogelijkheid is tot horizontale schaalbaarheid, ook is er de mogelijkheid om caching aan te zetten en een integratie met Memcache is ondersteund. 

\section{Selectie en uitwerking van de testsoftware}
De testen zijn geïmplementeerd als een uitbreiding van YCSB\cite{cooper2010benchmarking} omwille van verschillende redenen. Allereerst is de broncode publiek beschikbaar onder Apache 2.0, YCSB is een uitgebreid systeem voor het uitvoeren van performantie benchmarking, dit op basis van het meten van de vertraging op een query voor verschillende DBMS's. Hierdoor heeft deze al een uitgebreide ondersteuning voor tal van DBMS's, waaronder al de gekozen systemen. Deze ondersteuning is nog verder geoptimaliseerd voor de gekozen systemen zodat er maximaal gebruik gemaakt wordt van de functionaliteiten van elk systeem. Een concreet voorbeeld: bij het opstellen van de scan queries wordt rekening gehouden met het benodigd aantal records wat standaard in YCSB niet gebeurt bij het uitvoeren op een relationele database. 

De 2 testen, beschikbaarheidstest en consistentie test, worden op verschillende manieren geïmplementeerd, naar de leidraad van sectie \ref{sec:testenvandesystemen}.  

\paragraph{Beschikbaarheidstest} De beschikbaarheidstest wordt geïmplementeerd door middel van \textit{event support}, hiermee kan er op vooraf gedefinieerde momenten een bepaald Unix commando uitgevoerd worden. De configuratie gebeurt met behulp van een XML bestand met de parameters van \ref{table:beschikbaarheidinput} in bijlage, de output komt in het logbestand met de elementen van tabel \ref{table:beschikbaarheidoutput} in bijlage. 

Met behulp van deze uitbreiding zullen de beschikbaarheidstesten nadien uitgevoerd kunnen worden. Er zal gekeken worden naar de verandering in vertraging op een query waarmee kan bekeken worden of het systeem nog beschikbaar is. 

\paragraph{Consistentie testen} Voor de consistentie testen is er een extra module geïmplementeerd die het gedrag van sectie \ref{sec:testenvandesystemen} uitvoert. In deze uitwerking leest de schrijver niet zijn eigen data, al zou dit eenvoudig mee geïmplementeerd kunnen worden. Dit is niet getest omdat het niet nodig was in deze testen. De testen kunnen uitgebreid geconfigureerd worden om enkel te testen wat nodig is: een overzicht van de configuratie parameters is te vinden in tabel \ref{table:consistentieinput}. Voor elke uitgevoerde query, wordt een record aangemaakt met de data van tabel \ref{table:consistentieuitvoer} in bijlage. 

De code van deze testen is beschikbaar op GitHub onder  \url{https://github.com/thuys/YCSB-Implementation}. 

\section{Installatie en opstelling van de DBMS's en YCSB}
Het uitvoeren van de testen vereist het installeren van het verschillende instanties en de configuratie van de verschillende DBMS's. Voor het uitvoeren van de verschillende testen is het slechts nodig om het systeem een enkele keer op te zetten. Maar om de testen eenvoudiger te kunnen uitvoeren op verschillende infrastructuren en andere gebruikers de resultaten te laten controleren, is de installatie en configuratie van het systeem geautomatiseerd. 

De automatisatie gebeurt met het Integrated configuration Management Platform (IMP) beschreven in \cite{KULeuven-453199}. Dit modulair framework is uitgebreid met de 3 DBMS's en YCSB waardoor de configuratie als een declaratief gewenste staat wordt uitgedrukt. IMP zal deze staat toepassen op de verschillende systemen bij het uitrollen. 

Een uitgebreider bespreking van de uitwerking in IMP kan gevonden worden in bijlage \ref{chap:AppendixUitwerkingIMP} met het domeindiagram van het systeem, uitleg en voorbeeldcode. 

Voor de uitvoering van de testen, is er voor elk DBMS gekozen voor een minimaal aantal instantie dat datadistributie én replicatie ondersteunt. Voor de laatste eigenschap zou de data beschikbaar moeten blijven bij het uitvallen van 1 server na de overgangsperiode. In de testen is er enkel gefocust op het uitvallen van dataservers, niet naar configuratieservers. De configuratieservers zijn enkel minimaal opgezet omdat deze online blijven tijdens de testen.  

De opstelling van de systemen is getoond in figuur \ref{fig:deployment-testomgeving}, elke uitrol van de systemen zal in meer detail besproken worden nadat de testinfrastructuur is besproken.  

\begin{figure}[htbf]
\centering
\subfigure[Deployment van HBase met 5 instanties. ]{\label{fig:HBase-deployment}\includegraphics[width=0.55\textwidth]{img/HBase-deployment}}
\subfigure[Deployment van Pgpool-II met 3 instanties. ]{\label{fig:pgpool-deployment}\includegraphics[width=0.35\textwidth]{img/Pgpool-II-deployment}}
\subfigure[Deployment van MongoDB met 6 instanties. ]{\label{fig:MongoDB-deployment}\includegraphics[width=0.65\textwidth]{img/MongoDB-deployment}}
\subfigure[Deployment van de testomgeving met 2 YCSB instanties. ]{\label{fig:YCSB-deployment}\includegraphics[width=0.25\textwidth]{img/YCSB-deployment}}
\caption{Deployment van de verschillende DBMS's en de testomgeving.}\label{fig:deployment-testomgeving}
\end{figure}

De testinfrastructuur is een IaaS (Infrastructure as a Service) gebaseerd op OpenStack\footnote{https://www.openstack.org/}. De infrastructuur bestaat uit 3 Dell R610 en R620 servers met een totaal van 196GB RAM, 44 fysische CPU's (88 met hypertreading), verbonden met een Gigabit switch. Deze infrastructuur is gedeeld met andere gebruikers. Elke instantie heeft 2 virtuele CPU's, 4GB RAM en 50GB schijfruimte. De instanties worden verdeeld over de verschillende servers. De netwerkinfrastructuur heeft een gemiddelde ping van 0.4ms naar elke node ($\sigma = 0.2$ bij 10 000 ping's). 

\paragraph{HBase} Voor HBase wordt de data standaard 3 maal gerepliceerd en zijn er voor datadistributie dus 4 data instanties nodig. Elk van deze instanties hebben eenHBaseRegionServer en Hadoop datanode. Daarnaast zijn er nog een HMaster, Zookeeper en Hadoop namenode nodig die samen op een enkele instantie worden uitgerold. In het totaal zijn er 5 instanties. Een overzicht van de infrastructuur getoond in figuur \ref{fig:HBase-deployment}. De installatie en configuratiebestanden kunnen gevonden worden op \url{https://github.com/thuys/hbase}. 

\paragraph{Pgpool-II} Bij Pgpool-II is er ondersteuning voor horizontale schaalbaarheid in de parallel mode maar dit is niet getest. Om deze reden is er enkel replicatie toegepast waarvoor er 3 instanties zijn: een Pgpool-II instantie en twee PostgreSQL instanties. De configuratie van deze instanties zijn standaard met uitzondering van de activatie van de Write-Ahead-Log van PostgreSQL en de activatie van de replicatie mode in Pgpool-II. Een overzicht van de infrastructuur is getoond in figuur \ref{fig:pgpool-deployment}. De installatie en configuratiebestanden kunnen gevonden worden op \url{https://github.com/thuys/postgresql}.

\paragraph{MongoDB} MongoDB heeft ondersteuning in replicatie en datadistributie. Voor het beschikbaar zijn van de data bij het uitzetten van een enkele instantie, zijn er 3 MongoDB datanodes nodig in een replicaset. De data wordt verdeeld over 2 replicaset met behulp van sharding op basis van de hash van de key voor het opzoeken van de query. Omdat de toegangsserver en configuratie instanties niet veel resources innemen, zijn deze verspreid over de verschillende data instanties. Er zijn meerdere toegangsnodes geplaatst om de queries te verdelen naar verschillende toegangsnodes, bij de beschikbaarheidstesten zal een toegangsnode altijd beschikbaar blijven. In het totaal zijn er 6 instanties nodig. Deze zijn beschreven in \ref{fig:MongoDB-deployment} in bijlage. De installatie en configuratiebestanden kunnen gevonden worden op \url{https://github.com/thuys/mongodb}.

\paragraph{YCSB} YCSB kan naar meerdere instanties uitgerold worden. In deze testen is er gekozen om maar een enkele instantie uit te rollen om de testen eenvoudiger te kunnen uitvoeren.  Een overzicht is getoond in figuur \ref{fig:YCSB-deployment}. 


\section{Uitvoeren van de calibratie en testen}
Voor het uitvoeren van de volledige benchmarking dient eerst de verdeling van de type queries gespecificeerd worden, deze zijn voor alle verschillende systemen gelijk. Een overzicht van deze parameters kunnen gevonden worden in tabel \ref{table:calibratiequeries}. 40\% van de uitgevoerde queries past de database aan, er is dus een dynamische database. Bij het lezen wordt er de helft van de keren in batch gelezen met gemiddeld 50 records per bewerking. Tenslotte wordt er met een \textit{zipfian} verdeling gekozen om regelmatig dezelfde records te lezen waardoor de data aan het DBMS uit cache gelezen kan worden. 

Voor later de data optimaal te kunnen analyseren, wordt er elke second de gemiddelde vertraging gelogd voor elk type query. 

\paragraph{Calibratie testen} Voor de calibratie van de omgeving zijn er 2 soorten testen gedraaid, de parameters voor het aantal connecties kunnen gevonden worden in tabel \ref{table:calibratiegebruikers}. De parameters voor het aantal queries per second zijn te vinden in tabel \ref{table:calibratiequeriesperseconde}. In dit geval is het aantal gebruikers bepaalt door de uitkomst van de calibratie van het aantal gebruikers. 

\begin{table}[htb!]
	\centering
	\begin{tabular}{l| l }
	\textbf{Naam} & \textbf{Waarde} \\
	\hline
	Aantal velden & 10 (1 key veld) \\
	Record grootte & 1KB (100byte/veld) \\
	Lees alle velden & true \\
	Invoeg queries (\textit{insert}) & 20\%\\
	Lees queries (\textit{select}) & 40\%\\
	Aanpas queries (\textit{update}) & 20\%\\
	Scan queries (\textit{scan}) & 20\%\\
	Opvraag verdeling & zipfian (\textit{bepaalde records worden} \\
	& \textit{veel gelezen, andere weinig}) \\
	Maximale scan grootte & 100 \\
	Verdeling scan grootte & uniform \\
	\end{tabular}
	\caption{Overzicht van de query parameters}
	\label{table:calibratiequeries}
\end{table}

\begin{table}[htb!]
	\centering
	\begin{tabular}{l| l}
	\textbf{Naam} & \textbf{Waarde}  \\
	\hline
	Ingeladen records  & 300 000 \\
	Pauze & 50s \\
	Executie tijd & 600s \\
	Aantal gebruikers & 1, 2, 3, 4, 5, 7, 10, 15, \\
	& 20, 30, 40, 50, 75, 100\\
	\end{tabular}
	\caption{Calibratie: Overzicht van de parameters voor het testen van het aantal gebruikers}
	\label{table:calibratiegebruikers}
\end{table}

\begin{table}[htb!]
	\centering
	\begin{tabular}{l| l  }
		\textbf{Naam} & \textbf{Waarde}  \\
		\hline
		Ingeladen records  & 300 000  \\
		Pauze & 50s  \\
		Executie tijd & 600s \\
		Theoretisch aantal records & 20, 50, 100, 150, 200, 250, 300, 400, 500, \\
		per seconde  & 600, 700, 800,  900, 1000, 2000, 2500, 3000\\
	\end{tabular}
	\caption{Calibratie: Overzicht van de parameters voor het testen van het aantal records per seconde}
	\label{table:calibratiequeriesperseconde}
\end{table}

\paragraph{Beschikbaarheidstesten} Bij het uitvoeren van de testen op de beschikbaarheid van de verschillende systemen zijn de parameters in tabel \ref{table:beschikbaarheidstesten-parameters} gebruikt. Het aantal vooraf ingeladen records is op 300 000 geplaatst zodat zowel HBase als MongoDB aan sharding doen. De commando's voor het stoppen en starten van de systemen zijn te vinden in tabel \ref{table:beschikbaarheidstesten-commandos}.  Voor Pgpool-II is er een extra commando toegevoegd dat na het herstarten van de systemen wordt uitgevoerd. Dit komt omdat er geen automatische recovery in Pgpool-II is. Tenslotte worden deze testen uitgevoerd op al de datanodes, een overzicht hiervan met de overeenkomstige service is te vinden in tabel \ref{table:beschikbaarheidstesten-nodes}. De testen voor MongoDB zijn uitgevoerd op replicaset 2, dit is de set zonder enige configuratie- en toegangsservers servers. Een enkele replicaset is voldoende omdat de verschillende replicasets dezelfde functie hebben, enkel andere data opslaan. 

\begin{table}[ht!]
	\centering
	\begin{tabular}{l| l }
		\textbf{Naam} & \textbf{Waarde}  \\
		\hline
		Ingeladen records  & 300 000 \\
		Pauze & 50s \\
		Executie tijd & 900s \\
		Opstart kost & 100s \\
		Stoppen & Op 300s \\
		Starten & Op 600s \\
	\end{tabular}
	\caption{Beschikbaarheidstesten: Overzicht van de parameters}
	\label{table:beschikbaarheidstesten-parameters}
\end{table}


\begin{table}[ht!]
	\centering
	\begin{tabular}{l| l l }
		\textbf{Naam} & \textbf{Instanties} & \textbf{Service naam} \\
		\hline
		HBase  & HB2, HB3, HB4, HB5 & hbase-regionserver \\
		MongoDB  & MDB4, MDB5, MDB6, & mongodb-dataserver\\
		Pgpool-II  & PG1, PG2 & postgresql \\
	\end{tabular}
	\caption{Beschikbaarheidstesten: Overzicht van de instanties naar figuur \ref{fig:deployment-testomgeving}}
	\label{table:beschikbaarheidstesten-nodes}
\end{table}

\paragraph{Consistentie testen} Voor de consistentie testen moeten de parameters van tabel \ref{table:consistentieinput} geconfigureerd worden, de parameters zijn te vinden in tabel \ref{table:consistentie-testen-parameters}. Deze test wordt uitgevoerd op HBase en MongoDB. 

Om de analyse van de gegevens eenvoudiger te maken is er bij MongoDB gekozen om de test enkel uit te voeren op een replicaset en niet op een volledige cluster. Er is de aanname dat het consistentievenster afhankelijk is van de lengte tot de gegevens beschikbaar zijn op al de verschillende instanties van een replicaset. Het testen van een cluster voegt zo extra complexiteit toe. Deze test zou in de toekomst ook uitgevoerd kunnen worden op een cluster maar is in dit geval niet gedaan.  

\begin{table}[htb!]
	\centering
		\begin{tabular}{l|c c }
			\textbf{Naam} & \multicolumn{2}{c}{\textbf{Waarde}} \\ 
		 	 & \textbf{HBase} & \textbf{MongoDB} \\ \hline
			Ingeladen records  & \multicolumn{2}{c}{300 000} \\
			Pauze & \multicolumn{2}{c}{50s} \\
			Executie tijd & \multicolumn{2}{c}{900s} \\	
			starttime & \multicolumn{2}{c}{30s} \\
			readThreads & 10 & 5\\ 
			consistencyDelayMillis & 30ms & 10ms\\ 
			newrequestperiodMillis & \multicolumn{2}{c}{500ms} \\ 
			readProportionConsistencyCheck & \multicolumn{2}{c}{50\%} \\ 
			updateProportionConsistencyCheck & \multicolumn{2}{c}{50\%} \\ 
			stopOnFirstConsistency & \multicolumn{2}{c}{True} \\ 
			maxDelayConsistencyBeforeDropInMicros & \multicolumn{2}{c}{300ms} \\ 
			timeoutConsistencyBeforeDropInMicro & \multicolumn{2}{c}{300ms} \\
		\end{tabular} 
	\captionof{table}{Consistentie testen: Overzicht van de parameters}
	\label{table:consistentie-testen-parameters}
\end{table}

\section{Verzamelen en analyse van de testresultaten}
De analyse van de data gebeurt aan de hand van de informatie die gelogd wordt tijdens de executie van de testen. Voor alle mogelijke testen maakt R-code de data visueel in verschillende grafieken. Voor elke test kan de data op een andere wijze voorgesteld worden. 

De uitleg en voorbeelden van deze grafieken zullen getoond worden bij het presenteren van de resultaten in het volgende hoofdstuk. 

De R code kan gevonden worden op GitHub \url{https://github.com/thuys/YCSB-R-Scripts}.  

\section{Conclusie}
In dit hoofdstuk is de vertaling gemaakt van een theoretisch testmodel tot de implementatie. Daarbij zijn keuzes gemaakt en is de configuratie voor de volgende testen vastgelegd. De code voor elk gedeelte is te vinden op Github zodat anderen de testen kunnen reproduceren en aanpassen. De installatie van de model DBMS's is geautomatiseerd zodat deze met slechts weinig kennis kunnen opgesteld worden. De testresultaten zullen tenslotte visueel worden voorgesteld zodat het eenvoudiger is om de data te verwerken. 

De grote lijnen van de testmethode uit hoofdstuk 2 zijn geïmplementeerd, maar bepaalde mogelijkheden zijn niet geïmplementeerd. Dit is onder meer het geval voor lezen na het schrijven in de consistentie test en controleren of een waarde beschikbaar is in andere instanties na het platleggen van een instantie in de beschikbaarheidstest. 

\chapter{Observaties}
In dit hoofdstuk worden de resultaten getoond van de testen die zijn uitgevoerd, met een bespreking van de speciale elementen. Er zullen nog geen redenen aangehaald of conclusies getrokken worden, dit wordt in het volgende hoofdstuk gedaan. 

De resultaten zullen besproken worden per testsoort: eerst de resultaten voor de calibratie daarna voor beschikbaarheid en tenslotte voor consistentie. De figuren bevinden zich in bijlage \ref{sec:figobservaties}

De ruwe testdata kan geraadpleegd worden op \url{https://github.com/thuys/YCSB-Testdata}. 

\section{Calibratie}

\paragraph{Aantal gebruikers}
De resultaten van de calibratietest voor het aantal gebruikers kunnen gevonden worden in figuur \ref{fig:calibratie-gebruikers-resultaat}. Op de x-as is het gemiddeld aantal queries per second getoond over een periode van 600s, op de y-as de gemiddelde vertraging. De verschillende punten stellen een aantal gebruikers voor die zijn aangegeven met het bijhorend getal. 

Het aantal gebruikers wordt zo gekozen dat het totale aantal queries zakt of voor een sterke groei in vertraging zorgt, dit zorgt voor de gegevens in tabel \ref{table:calibratie-gebruikers-resultaat}. Bij MongoDB is er voor een lage waarde van 15 gebruikers gekozen, in plaats van 50, de reden hiervoor is dat de variatie in de vertraging groter wordt bij meer gebruikers, wat de data van de overige testen potentieel minder stabiel maakt.  

\begin{table}[h!]
	\centering
	\begin{tabular}{l| l }
		\textbf{DBMS} & Aantal gebruikers \\
		\hline
		HBase & 50 \\
		MongoDB & 15\\
		Pgpool-II & 30\\
	\end{tabular}
	\caption{Calibratie: Aantal gebruikers per test voor de verschillende DBMS's}
	\label{table:calibratie-gebruikers-resultaat}
\end{table}

\paragraph{Aantal queries per seconde}
De resultaten voor de calibratietest voor het aantal queries per seconden kunnen gevonden worden in de figuren \ref{fig:calibratie-queriesperseconde-hbase}, \ref{fig:calibratie-queriesperseconde-mongodb} en \ref{fig:calibratie-queriesperseconde-pgpool-ii} voor respectievelijk HBase, MongoDB en Pgpool-II. Deze figuren tonen in de bovenste figuur de gemiddelde vertraging op een query afhankelijk van het aantal queries per seconde, naarmate het aantal queries toeneemt stijgt de vertraging, met uitzondering van een laag aantal queries. De onderste figuur toont op de y-as de verhouding tussen het eigenlijk aantal uitgevoerde queries per seconde t.o.v. het gevraagde aantal queries per seconde. Een fictief voorbeeld: bij het vragen van 100 queries/sec wordt er in de praktijk maar 60 uitgevoerd, dit zorgt voor een waarde van $0.6$. 

Met beide figuren samen, kan een matige belasting gekozen. Een matige belasting is een belasting waarbij de onderste figuur de waarde 1 zo dicht mogelijk benaderd en de vertraging nog niet te veel is gestegen t.o.v. van een lage belasting. De gekozen waarde zijn te vinden in tabel \ref{table:calibratie-queriesperseconde-resultaat}. 

\begin{table}[htb!]
	\centering
	\begin{tabular}{l| l }
		\textbf{DBMS} & Aantal requests per seconde \\
		\hline
		HBase & 600 \\
		MongoDB & 200\\
		Pgpool-II & 100\\
	\end{tabular}
	\caption{Calibratie: Aantal queries per seconde per test bij een matige belasting voor de verschillende DBMS's.}
	\label{table:calibratie-queriesperseconde-resultaat}
\end{table}



\section{Beschikbaarheidstest}
Bij de beschikbaarheidstesten kunnen de gegevens op verschillende manieren voorgesteld worden: de vertraging per query over de hele test, de vertraging tijdens het stoppen en starten van systemen of een vergelijking van de vertraging voor het stoppen (150-250s), na het herstarten (700-800s) en tussen het stoppen en starten(400-500s). 

Voor elk van de systemen is voor al de acties op de verschillende instanties data in voorhand, maar slechts enkele grafieken zullen getoond worden. Al de grafieken kunnen gevonden worden op GitHub op de link gegeven in het begin van het hoofdstuk. 

Een punt op de grafiek stelt de gemiddelde vertraging van 1 seconde voor, de lijn het gemiddelde over 10 seconden. 

\paragraph{HBase}
Bij HBase zijn er verschillende reacties op het stopzetten van een node. Bij het zacht stoppen van een instantie, is er een onderbreking van gemiddeld ongeveer 20 seconde in de testen. Daarna kunnen er nog verhogingen in de queries af en toe optreden. Zie figuur \ref{fig:beschikbaar-hbase-soft}. 

Bij de netwerk onderbreking is er in de testen een onderbreking van gemiddeld ongeveer 100 seconden. Daarna is het terug stabiel. Zie figuur \ref{fig:beschikbaar-hbase-drop}

Bij een harde stop is er een combinatie van de netwerk onderbreking én is het af en toe zo dat de volledige periode geen queries mogelijk zijn. Zie figuur \ref{fig:beschikbaar-hbase-drop} en \ref{fig:beschikbaar-hbase-hard}

Tijdens de onderbreking (400-500s), is er geen significante verandering in de vertraging van de uitgevoerde queries gemeten (t.o.v. 150-250s en 700s - 800s). 


\paragraph{MongoDB}
Ook bij MongoDB zijn er verschillende reacties op het stopzetten als men de queries onder standaard configuratie uitvoert. In het geval van zacht of hard stoppen is er geen verschil in de reactie, bij 2/3 van de keren is er geen verschil merkbaar, bij 1/3 van de keren is er tijdelijke verhoging van de vertraging, een voorbeeld toont dat de scan operatie voor 2 seconden uitgesteld wordt. Figuur van het overzicht: \ref{fig:beschikbaar-mongodb-soft} met een zoom naar de stop \ref{fig:beschikbaar-mongodb-soft-zoom}.

Bij het onderbreken van het netwerk is er in bepaalde geen significante verandering, op andere momenten is een gedrag soortgelijk aan dat bij een zachte stop te merken. In andere gevallen is het zo dat er geen queries mogelijk zijn gedurende de volledige netwerk onderbreking. Zie figuur \ref{fig:beschikbaar-mongodb-drop}. 

Tijdens de onderbreking (400s- 500s), is er geen significante verandering in de vertraging van de uitgevoerde queries gemeten (t.o.v. 150-250s en 700s - 800s). 
 


\paragraph{Pgpool-II} Bij Pgpool-II is een green verschil tussen een harde of zacht stop. In beide gevallen is er tijdelijk een onderbreking van al de queries, er is een verhoogde vertraging van ongeveer 2 seconden, een voorbeeld bevindt zich in figuur \ref{fig:beschikbaar-pgpool-soft}. 

Bij een netwerk onderbreking, zijn er enige tijd geen queries mogelijk en na 30 seconden was de onderbreking over in de testen. Een voorbeeld bevindt zich in figuur \ref{fig:beschikbaar-pgpool-netwerk}.  

Tijdens de onderbreking is er een verandering naar schrijfbewerkingen toe, deze nemen significant minder tijd in beslag, dit geldt niet voor leesbewerkingen. Voorbeelden uit de testen bevinden zich voor beiden in figuren \ref{fig:beschikbaar-pgpool-boxplot-read} en \ref{fig:beschikbaar-pgpool-boxplot-write}. 

De herstel van een server na deze opnieuw online gebracht hebben, lukt slechts in zeldzame keren na deze te hebben opstarten. Enkel als alle connecties zijn, verbroken lukt het herstel. 

\section{Consistentie test}
Voor de consistentie testen worden er empirische verdelingsfuncties gebruikt. Dit zijn functies waarbij op de y-as het percentage van de waarden kleiner dan x staat aangegeven.

Voor de consistentie testen wordt er op de x-as de start- en/of stoptijdstippen van de verschillende soorten getoond. Het verschil tussen de y-waarde van de start- en stoptijdstippen geeft aan hoeveel queries er op dat moment uitgevoerd worden. De startmomenten van een lezer zijn de eerste keer dat deze de correcte data leest. 

\paragraph{HBase}
Bij HBase is er geen verschil tussen het invoegen of aanpassen van data naar consistentie, dit zijn dezelfde queries. Daarnaast zijn er geen configuratie mogelijkheden voor het lezen of schrijven van data naast het in- of uitschakelen van de caches aan de gebruikerskant, deze uitgeschakeld zijn al de testen. 

Figuur \ref{fig:consistentie-hbase-start} toont een overzicht van de verschillende starttijdstippen voor het lezen van consistente data. Figuur \ref{fig:consistentie-hbase-R1} toont de start- en eindtijdstippen voor lezer 2 naast deze voor de schrijver. De maximale waarde van de x-as is zo gekozen dat voor elke dataset minstens 99\% van de data getoond is. 



\paragraph{MongoDB} Bij MongoDB zijn er 5 soorten lees- en 5 soorten schrijfconfiguraties mogelijk. Na de testen bleek het dat secondarypreferred gelijk was aan secondary en is niet getoond in de grafieken om deze reden. 

Voor elk van de 5 mogelijke schrijvacties, is een overzicht gegeven in figuren \ref{fig:consistentie-mongodb-all} en \ref{fig:consistentie-mongodb-R2}. In de eerste figuur, is de data van al de lezers gecombineerd, bij de tweede wordt er enkel naar lezer 2 gekeken. De maximale waarde van de x-as is zo gekozen dat voor elke dataset minstens 97\% en 99\% van de leesdata is getoond inor respectievelijk figuur \ref{fig:consistentie-mongodb-all} en \ref{fig:consistentie-mongodb-R2}. Er zijn telkens de update queries getoond met een vergelijking tussen de insert en update bij majority. Ook uit de andere data blijkt dat deze niet significant verschillend zijn. 

Een vergelijking van de duur van de verschillende leesoperaties kan gevonden worden in figuur \ref{fig:consistentie-mongodb-all-mongodb-write} waar er minstens 90\% van de leesdata is getoond. 


\section{Conclusie}
De resultaten zijn verschillend voor de verschillende systeem, in het volgend hoofdstuk zullen de resultaten geanalyseerd worden en mogelijke verklaringen gezocht worden. 

\chapter{Analyse van de resultaten}
In dit hoofdstuk zullen de resultaten van het vorige hoofdstuk aanbod komen en de mogelijke oorzaken besproken worden. 

Daarnaast worden de verschillende systemen met elkaar vergeleken wat bepaalde keuzes . 
\section{Calibratie}
Over de calibratie testen valt in het algemeen niet veel af te leiden, deze testen zijn niet uitgevoerd op een volledig dezelfde infrastructuur, zo heeft Pgpool-II slechts 3 instanties t.o.v. 6 voor MongoDB, maar dat was ook niet het doel. 

Enkel op de stijgende variatie van de vertagingen in MongoDB zal dieper ingegaan worden. De reden hiervoor is een lezers/schrijving locking systeem op een gehele database\cite{mongodb-concurrency}. Hierdoor zorgt een leesactie voor de blokkering van alle schrijfactie op de database en vice versa. Naar mate er meer gebruikers zijn, kunnen er meer opeenvolgende schrijfoperaties zijn, dit zal de leesacties langer blokkeren. Maar indien alle gebruikers samen lezen, kan dit parallel gebeuren, een grotere variatie in de vertraging treedt hierdoor op. 

De netwerkinfrastructuur heeft een gemiddelde ping van 0.4ms naar elke node ($\sigma = 0.2$ bij 10 000 ping's). 

\section{Beschikbaarheidstest}
Bij de beschikbaarheidstesten blijkt er uit de resultaten dat de verschillende systemen een andere aanpak hebben genomen. Deze zullen nu verder in detail besproken worden. Belangrijk  is dat er hier bewerking aan de basisbelasting uitgevoerd worden, waardoor er data van de verschillende datadistributies gelezen zal worden. 

\paragraph{HBase} Bij HBase heeft een bepaalde RegionServer de verantwoordelijkheid over een Regio voor een bepaalde tijd. Dit is een sessie dit door HMaster uitgedeeld wordt en bijgehouden wordt in Zookeeper. Deze sessie kan vroegtijdig beëindigd worden of er moet gewacht worden tot deze verlopen is, enkel op die momenten kan er een nieuwe RegionServer aangeduid worden. Dit zorgt voor een duidelijk verschil tussen een zachte stop, een harde stop of netwerk probleem. 

De duur van een sessie kan geconfigureerd worden in Zookeeper en staat standaard op 180 seconden. \cite{hbase-doc}. 

\subparagraph{HBase: Zachte stop} Bij een zachte stop, is er slechts af en toe sprake dat dit merkbaar is, de verklaring hiervoor is dat dit enkel wordt opgemerkt als de RegionServer die op dat moment verantwoordelijke is voor de Region wordt stopgezet. In de testen is dit niet zichtbaar omdat er verschillende opeenvolgende queries worden uitgevoerd. \\
Indien deze RegionServer wordt stopgezet, nemen de queries tijdelijk meer tijd in beslag.  Het terug online brengen van de server heeft geen invloed op de snelheid een query wordt uitgevoerd . Na het stopzetten van de RegionServer is er in bepaalde gevallen een verhoogde vertraging in beide leesoperaties (scan en lees). \\
Zodra er een herverdeling is van de Regions over de aanwezige Regionservers, verdwijnt deze verhoogde vertraging. 

\subparagraph{Netwerk onderbreking} Bij een netwerk onderbreking, worden de queries tijdelijk stopgezet en falen de queries in tussentijd. Deze onderbreking duurt significant langer dan in het geval van een zachte stop. Dit komt doordat de regio's pas kunnen toegewezen worden na het verlopen van hun sessie.  

\subparagraph{HBase: Harde stop} Bij het stopzetten van een instantie op de harde manier, zijn er 2 gedragen: het eerste is gelijk aan deze van een netwerk onderbreking. De andere laat pas opnieuw queries toe na het herstellen van het netwerk verkeer. In een manuele test bleek dit opgelost te zijn na het verbreken van de connectie en het opnieuw verbinden, maar de oorzaak waarom dit slecht af en toe voorkomt, is niet gevonden. 

\subparagraph{Herstel van de instantie} Het herstel van de server zal automatisch op een asynchrone manier gebeuren. Er valt ook te configureren hoeveel data er maximaal per seconde zal worden gesynchroniseerd. Het herstel is niet merkbaar voor de gebruiker bij de testen. 

\paragraph{MongoDB} Bij MongoDB is er tussen de leden van een Replicaset een heartbeat protocol. Indien er gedurende 10 seconden geen antwoord op een een heartbeat komt, wordt een server als offline bestempeld \cite{mongodb-manual}. Dit heeft opnieuw zijn invloed op de verschillende manieren om een server stop te stopzetten. 

\subparagraph{Zachte stop en harde stop} Bij een zachte of harde stop is er een kans van 1 op 3 dat het uitvallen van een instantie zichtbaar is, dit is te verklaren doordat enkel het uitschakelen van de primary een invloed zal hebben op de vertraging, in de standaard modus werd er enkel gelezen naar en geschreven van de primary. Nadien is er geen invloed bij de verschillende queries naar de vertraging. De reden dat een harde stop op dezelfde manier reageert kan verklaard worden op 2 wijzen: bij een harde stop wordt de sessie nog altijd vrijgegeven, of er zijn verkiezingen voor een nieuwe primary voordat de sessie van de oude primary is afgelopen. 

\subparagraph{Netwerk onderbreking} Bij een netwerk onderbreking zou het te verwachten zijn dat na 10 seconde een primary zou veranderen. Onder de aangelegde belasting blijkt dat de database heel de tijd onbeschikbaar tot het opnieuw beschikbaar maken van het netwerk. Bij het manueel testen blijkt dat er dezelfde foutmelding gegeven wordt als bij het stoppen van de database, maar dat de data onbeschikbaar is. Het volstaat om de verbindingen af te sluiten en opnieuw aan te maken om het probleem op te lossen.

\subparagraph{Herstel van de instantie} Het herstel van de server zal automatisch op een asynchrone manier gebeuren. Dit is niet merkbaar voor de gebruikers. 

\paragraph{Pgpool-II} Bij Pgpool-II wordt er bij het hebben van een connectie naar Pgpool-II, de connecties naar de verschillende PostgreSQL instanties gecontroleerd. Bij het uitvallen van een instantie en opnieuw opstarten terwijl er geen gebruiker verbonden is met Pgpool-II, zal dit niet opgemerkt worden. Daarnaast zijn er wel verschillende reacties op de geteste scenario's. 

Een vereiste bij het herstellen van een instantie is dat er op dat moment geen enkele gebruiker actief is. 

\subparagraph{Zachte stop} Bij een zachte stop van een data instantie worden alle verbinden met Pgpool-II verbroken, nadien kan er terug verbonden worden met Pgpool-II. In deze omgeving gaan nadien de verschillende schrijfoperaties sneller omdat deze niet meer gerepliceerd moeten worden, bij een grote hoeveelheid data instanties zal dit effect kleiner worden.  

\subparagraph{Harde stop} Een harde stop reageert hetzelfde als een zachte stop, dit omdat ook hier de connecties onmiddellijk verbroken zijn, de data instantie zal antwoorden dat er geen service op de poort aan het luisteren is. 

\subparagraph{Netwerk onderbreking} Bij een netwerk onderbreking is er een ander gedrag, de queries wachten op een antwoord maar krijgen dit niet. Hierdoor wordt er gewacht op de time-out die standaard 30 seconde is. Na deze tijd worden connecties verbroken, kan er opnieuw verbonden worden en kunnen queries opnieuw uitgevoerd worden. 

\subparagraph{Vermindering van leesvertraging} De reden tot de vermindering van leesvertraging is te vinden in de manier dat Pgpool-II de replicatie van de queries doet. Deze zullen eerst op de master uitgevoerd worden en vervolgens op de slaves. Bij het wegvallen van een instantie is er nog maar een enkele data server over, hierdoor duurt een schrijfactie maar half zo lang. De leesacties duren ongeveer even lang aangezien dezelfde acties nog steeds genomen worden. 

\subparagraph{Herstel van de instantie} Bij het opnieuw inschakelen van een instantie dient in Pgpool-II het herstel handmatig in gang gezet te worden. De data zal van de master naar de instantie gesynchroniseerd worden. In het geval van een grote achterstand zal dit merkbaar zijn omdat het proces aan maximale snelheid wordt uitgevoerd; een grote belasting op de CPU, harde schijf en het netwerk kunnen dus voorkomen. Om het herstel te voltooien moeten alle connecties naar Pgpool-II op een gegeven moment gesloten worden. In de testen die werden uitgevoerd waren er steeds actief en hierdoor slaagde het herstel niet. 

\paragraph{Conclusie} Hoewel er verschillende reacties zijn tussen HBase en MongoDB, ligt de interne werking vrij dicht bij elkaar, de status van beiden wordt permanent opgevolgd. Bij MongoDB gebeurt dit door de data instanties zelf en kan de parameter niet aangepast worden. Bij HBase is er een extern systeem voor gebruikt waarbij de parameter geconfigureerd worden. Pgpool-II heeft een heel ander systeem door enkel de instanties te controleren op het moment dat er een verbinding is. 
Daarnaast ondersteunt Pgpool-II ook niet de automatische herstel en komt de handmatige herstel niet tot voltooiing onder constant gebruik, hiervoor zijn beide andere systemen automatischer. Een overzicht van het gedrag bij het stoppen en starten van een instantie , bevinden zich in tabel \ref{table:beschikbaarheid-stop-resultaat} en \ref{table:beschikbaarheid-herstel-resultaat}.

\begin{table}[htbp]
  \centering
    \begin{tabular}{l | lll}
          & Zachte stop & Harde stop & Netwerk stop \\
    \hline
    \multirow{2}{*}{HBase} & Enkele seconden & Tiental seconden & Tiental seconden \\
    & & of onbeperkt&  \\
    \multirow{2}{*}{MongoDB} & 1/3 van de gevallen, & 1/3 van de gevallen, & Enkele seconden tot \\
    & enkele seconden & enkele seconden & Onbeperkt\\
    Pgpool-II & Enkele seconden & Enkele seconden & 30 seconden \\
    \end{tabular}%
    \caption{Beschikbaarheid: Overzicht van de reacties bij het stoppen van een instantie }
  \label{table:beschikbaarheid-stop-resultaat}%
\end{table}

\begin{table}[htbp]
  \centering
    \begin{tabular}{l|l}
          & Automatisch herstel \\
    \hline
    HBase & Ja \\
    MongoDB & Ja \\
    Pgpool-II & Nee \\
    \end{tabular}%
      \caption{Beschikbaarheid: Overzicht van de ondersteuning van automatisch herstel}
  \label{table:beschikbaarheid-herstel-resultaat}%
\end{table}%



\section{Consistentie test}
\paragraph{HBase} HBase garandeert strikte consistentie op een enkel record en hoe deze garantie tot uitvoering wordt gebracht, is duidelijk zichtbaar in figuur \ref{fig:consistentie-hbase-R1}. Een lees query wordt namelijk op wacht gezet tot de schrijf query voltooid is, dit valt af te lezen doordat de lijn van het stoppen met schrijven een hogere waarde heeft als het stoppen met lezen en dit zo de hele tijd is, daarnaast volgt het stoptijdstip van de leesbewerking deze van de schrijfwerking. Tenslotte kan ook de data in meer detail bekeken worden en is dit zichtbaar. In figuur \ref{fig:consistency-hbase-uitleg} wordt het lees- en schrijfmodel van HBase uitgelegd met de uitleg van Lars Hofhansl\cite{hbase-acid}. De combinatie van een enkele HRegionServer voor een record en het gebruiken van locks, zorgt ervoor dat atomaire acties op een enkele record succesvol afgedwongen kunnen worden.     

Uit de testresultaten blijkt dat indien de leesbewerking te snel verstuurd wordt, er nog geen blokkering van de bewerking zal plaats vinden. Het percentage van de queries dat vanaf de eerste keer al de juist data zal lezen, bevindt zich in tabel \ref{table:consistentie-hbase-correct}.

\begin{table}
\centering
\begin{tabular}{l| l l}
	\textbf{Lezer} & \textbf{Starttijdstip (ms)} & \textbf{Percentage eerste keer} \\
	\hline
	1 &  0 ms & 2.6\%\\
	2 &  3 ms & 68\%\\
	3 &  6 ms & 90\%\\
	4 &  9 ms & 93\%\\
	5 &  12 ms & 94\%\\
	6 &  15 ms & 96\%\\
	7 &  18 ms & 96\%\\
	8 &  21 ms & 96\%\\
	9 &  24 ms & 97\%\\
	10 & 27 ms & 97\% 
\end{tabular}
\caption{Consistentie: Percentage van de queries dat van de eerste keer de juiste data leest voor HBase. Het gemiddelde verschil tussen het starten en stoppen van het lezen op een willekeurig record is ongeveer 6ms. }
\label{table:consistentie-hbase-correct}
\end{table}

\begin{figure}[tb!]
	\begin{minipage}{0.5\textwidth} 
	\textbf{Schrijven}
	\begin{enumerate}
	\item Lock de rij(en), om te beschermen tegen concurrente schrijfacties. 
	\item Haal het huidige schrijfnummer op
	\item Voeg aanpassingen toe aan WAL (Write Ahead Log)
	\item Pas aanpassing toe op de Memstore (cache geheugen)
	\item Commit de transactie, m.a.w. zet het leespunt op het nieuwe schrijfnummer
	\item Unlock de rijen
	\end{enumerate}
	\end{minipage} \hfill
	\begin{minipage}{0.3\textwidth} 
	\textbf{Lezen}
	\begin{enumerate}
	\item Open de lezer
	\item Ga naar het huidige leespunt
	\item Filter al de KeyValues paren met schrijfnummer > leespunt
	\item Sluit de lezer
	\end{enumerate}
	\end{minipage}
	\caption{HBase: Het vereenvoudigde lees- en schrijfmodel voor strikte consistentie in HBase naar Lars Hofhansl\cite{hbase-acid}}\label{fig:consistency-hbase-uitleg}
\end{figure}

\paragraph{MongoDB} MongoDB biedt strikte consistentie aan als er van de primary gelezen wordt maar er zijn ook andere schrijf- en leesmethodes. Een verschil met HBase is dat het bij alle mogelijke lees- en schrijfmethodes mogelijk is om de nieuwe data al te lezen vooraleer de schrijfbewerking beëindigd is. Een schrijf query wacht op de server nog na het schrijven en vrijgeven van zijn schrijf lock. Een verklaring is hiervoor niet gevonden.

Een analyse van de data uit figuur \ref{fig:consistentie-mongodb-all}, kan tonen hoeveel percent kans er is dat data consistent gelezen zal worden vanaf een bepaald tijdstip.  Voor de waarden van 0, 2, 4, 6 en 8 ms is dit berekend, respectievelijk lezer 1 tot 5,  en kunnen de waarden teruggevonden worden in \ref{table:consistentie-mongodb-correct}. Het is opvallend dat met uitzondering van het lezen van de dichtstbijzijnde, er geen groot verschil is tussen de verschillende percentages van verschillende schrijfoperaties. De schrijf configuraties geven dus geen garanties tijdens het uitvoeren maar enkel na de voltooiing van de schrijfoperatie.  Na verder onderzoek blijkt het verschil bij nearest niet significant te zijn doordat er net iets meer lezers een primary op dat moment als dichtstbijzijnde hadden. 

Uit tabel \ref{table:consistentie-mongodb-inconsistency} blijkt dat het in MongoDB niet is gegarandeerd dat als een lezer de nieuwe waarde leest, dat al de overige lezers dat ook zullen doen. In dit geval was het schrijven nog niet voltooid en een bepaalde lezer leest de nieuwe data al, een latere bewerking leest de oude waarde nog. Dit kan verklaard worden doordat het verschillende servers zijn waarop gelezen wordt. Maar aangezien de MongoDB driver periodiek controleert welke server het dichtste bij is, kan dit juist tussen deze 2 bewerkingen gebeuren als men niet de leesgarantie op primary zet. In dit geval is er géén garantie op monotone leesbewerkingen. 

Aangezien er in de testomgeving een uniforme netwerkvertraging is naar alle instanties, volgt de data de veronderstelling dat de dichtstbijzijnde node in iets minder 1/3 van de gevallen een primary is en iets meer dan 2/3 een secondary. Met 5\% afwijking is het moeilijk te stellen dat deze significant is. 

Tenslotte hebben primary en primary-preferred in deze testen geen significante verschillen. Dit komt omdat de primary heel de tijd beschikbaar is. 

\begin{table}
\centering
\begin{tabular}{l | l l l l}
Lezer & Start lezen (ms) & Stop lezen (ms) & Gelezen waarde & Correct? \\
\hline
\multirow{2}{*}{1} & 2,200 & 3,213 & 12553\textbf{3}813315 & Nee\\
 & 13,426 & 14,279 & 12553\textbf{4}813315 & Ja \\
 \multirow{2}{*}{3} & 17,458 & 18,834 & 12553\textbf{3}813315 & Nee\\
 & 29,063 & 29,897 & 12553\textbf{3}813315& Ja \\
\end{tabular}
\caption{Consistentie: Ruwe data van MongoDB test waarbij inconsistente data wordt gelezen na het lezen van consistente data op verschillende lezers met het lezen via nearest en schrijven via fsync\_safe}
\label{table:consistentie-mongodb-inconsistency}
\end{table}

\begin{table}[ht]
\centering
\begin{tabular}{L{2.3cm}| L{2.3cm} L{2.3cm} L{2.3cm} L{2.3cm}}
& \parbox[t]{2.3cm}{nearest} & \parbox[t]{2.3cm}{primary} & \parbox[t]{2.3cm}{primary- \newline preferred} & \parbox[t]{2.3cm}{secondary }\\ 
  \hline
safe & \parbox[t]{2.3cm}{28, 69, 89, 91, 92} & \parbox[t]{2.3cm}{80, 98, 98, 99, 99} & \parbox[t]{2.3cm}{74, 99, 99, 99, 99} & \parbox[t]{2.3cm}{0, 65, 83, 85, 88 }\\ 
  normal & \parbox[t]{2.3cm}{24, 68, 87, 89, 92} & \parbox[t]{2.3cm}{72, 99, 100, 100, 100} & \parbox[t]{2.3cm}{75, 98, 98, 98, 98} & \parbox[t]{2.3cm}{0, 69, 85, 89, 92 }\\ 
  fsync\_safe & \parbox[t]{2.3cm}{28, 73, 87, 90, 90} & \parbox[t]{2.3cm}{68, 96, 98, 98, 98} & \parbox[t]{2.3cm}{78, 97, 98, 98, 98} & \parbox[t]{2.3cm}{0, 66, 80, 85, 86 }\\ 
  replicas\_safe & \parbox[t]{2.3cm}{24, 74, 87, 88, 91} & \parbox[t]{2.3cm}{75, 98, 99, 99, 99} & \parbox[t]{2.3cm}{79, 98, 98, 98, 98} & \parbox[t]{2.3cm}{1, 67, 84, 87, 89 }\\ 
  majority & \parbox[t]{2.3cm}{26, 77, 91, 91, 92} & \parbox[t]{2.3cm}{73, 98, 99, 99, 99} & \parbox[t]{2.3cm}{77, 99, 99, 99, 100} & \parbox[t]{2.3cm}{0, 61, 82, 85, 89 }\\ 
\end{tabular}
	\caption{Consistentie: Percentage van de queries dat van de eerste keer juist de data leest bij 0ms, 2ms, 4ms, 6ms en 8ms voor MongoDB. Met als rijen de verschillende schrijf types en als kolommen de verschillende lees types. De gemiddelde vertraging op een onafhankelijke leesoperatie is 1ms. }
	\label{table:consistentie-mongodb-correct}
\end{table}

\paragraph{Conclusie} Beide database systemen bieden strikte consistentie aan maar hebben een verschillende uitwerking hiervan: bij HBase worden de leesoperaties uitgesteld tot de volledige voltooiing van de schrijfoperatie, bij MongoDB zal de data al vroeger beschikbaar zijn. Beide systemen zijn \textit{session} consistent en \textit{read-your-own-write} consistent indien er op een primary wordt gelezen voor MongoDB. 

\textit{Session}, \textit{read-your-own-write}, \textit{casual} en \textit{monotonic} consistentie zijn niet gegarandeerd in MongoDB indien er niet gelezen wordt op een primaire. De MongoDB driver kan op ieder moment een andere server kiezen in deze gevallen en kan dus nog oude data lezen. HBase heeft deze garanties wel. 

Het falen van de primary tijdens de schrijfoperaties kan de consistentie garanties beïnvloeden, een nieuwe primary kan verkozen worden die de data nog niet had ontvangen. Maar een gebruiker zou de data al wel van de oude primary gelezen kunnen zijn, in dit geval faalt hier de stikte consistentie. Dit gedrag is wel niet getest en bevestigd. HBase heeft deze situatie niet door de keuze om de leesbewerking te verlengen, een gebruiker dient dus langer te wachten op zijn data. 

\section{Conclusie}
De drie systemen hebben verschillende aanpak naar beschikbaarheid en consistentie. Pgpool-II is het minst geavanceerd systeem door geen automatisch herstel te ondersteunen, maar door de centrale aanpak van de toegangsnode gebruikt dit systeem geen netwerk verkeer als het niet wordt gebruikt. 

MongoDB is een systeem dat weinig configureerbaar is naar het gedrag bij het falen van een instantie, daarentegen zijn er een verschillende configuratiemogelijkheid naar het lees- en schrijfgedrag. Alhoewel het strikte consistentie zegt aan te bieden, kunnen er vraagtekens bij gezet worden. Enkel als er gelezen wordt van de primary, zal er strikte consistentie zijn, waar het nog onduidelijk is welke garanties er zijn bij het falen van een primary. Daarnaast is het in normale situaties mogelijk om de nieuwe data snel te kunnen lezen. 

HBase is met behulp van Zookeeper configureerbaar naar het gedrag bij falen van een enkele instantie, de onbeschikbaarheidsperiode kan verkleind of vergroot worden. De consistentie garanties van HBase zijn strikt voor een enkel record maar dit komt wel voor de prijs dat een leesactie uitgesteld wordt indien er een de schrijfactie op dat record uitgevoerd wordt. 
\chapter{Conclusie}\label{sec:conclusie}
Deze thesis heeft de kloof verkleint tussen de informatie die nodig en beschikbaar is voor het selecteren van een DBMS. Dit is gebeurd door het ontwikkelen een nieuwe benchmarking tool om consistentie- en beschikbaarheidseigenschappen te onderzoeken. Daarnaast werd deze tool toegepast om drie verschillende voorbeeldsystemen te onderzoeken en vergelijken. Al deze software is beschikbaar en eenvoudig bruikbaar voor andere gebruikers door de hulp een automatische installatie en configuratie van de benchmarking tool en de DBMS's. 

In dit hoofdstuk zullen eerst de uitdagingen besproken worden die overwonnen werden bij deze thesis. Dit wordt gevolgd met een overzicht voor potentieel verder onderzoek. Tenslotte wordt de thesis afgesloten met een samenvatting van de belangrijkste bijdragen. 

\section{Uitdagingen}
\paragraph{Handmatig opstellen van de DMBS's} De grootste uitdaging van de thesis was om de drie verschillende DBMS's handmatig te installeren en configureren tot een feilloos werkend systeem. Dit was een uitdaging omwille van verschillende redenen. 

Allereerst bestaan alle drie de systemen uit veelvoud van services en verschillende services gebruiken een andere configuratie methode. Bij HBase zijn er niet minder dan 5 verschillende services die elk geconfigureerd worden door middel van bestanden. Bij MongoDB verloopt de configuratie op zijn beurt via de terminal. 

Daarnaast zijn deze systemen nog volop in ontwikkeling, hierdoor is de documentatie vaak niet al te uitgebreid en zijn de foutmeldingen vaag. Dit was onder andere het geval bij Pgpool-II waarbij er enkel documentatie was voor Debian, maar de testinfrastructuur werkt met Fedora. Onder andere het gebruik van de beveiligingsmodule SELinux heeft de correcte werking van het herstel van Pgpool-II lange tijd gehinderd. 

\paragraph{Automatische installatie en configuratie van DBMS's} Als een systeem manueel kan opgezet worden, betekent dit niet dat er een onmiddellijke vertaling is naar het automatisch opstellen met behulp van IMP, hiervoor zijn er twee redenen. 

Allereerst kan het zijn dat een lijn in de terminal niet hetzelfde effect heeft dan wanneer deze in een bashscript wordt uitgevoerd. Het commando kan afhankelijk zijn van de huidige situatie van het DBMS. Bij MongoDB was het nodig om eerst de huidige configuratie te weten van de cluster voor het toevoegen of verwijderen van servers of replicaset's. 

Daarnaast kon IMP op het moment van de ontwikkeling nog geen afhankelijkheden aan tussen verschillende servers. Onder andere in MongoDB zorgde dit voor uitdagingen. Een replicaset kan bijvoorbeeld maar geïnitialiseerd worden wanneer alle servers online zijn. Met behulp van de thesis van Harm De Weirdt\cite{thesisHarm} is dit nu wel mogelijk in IMP. Als zijn methode toegepast wordt, zou de uitrol eenvoudiger kunnen verlopen.  

Hoewel de automatisatie de nodige tijd en moeite heeft gekost, heeft dit zijn plaats in deze thesis. Allereerst is het voor andere gebruikers eenvoudiger om de systemen op te zetten en de testen uit te voeren, dit was ook het geval voor mijzelf. Tijdens de ontwikkeling van de benchmarking tool was het af en toe nodig om al de servers te resetten. Het opnieuw opzetten van de infrastructuur ging veel vlotter, ook kon er eenvoudiger geëxperimenteerd met verschillende configuraties. 

\paragraph{Uitvoeren van de benchmarking tool} Een laatste uitdaging was het ontwikkelen van de benchmarking tool en de configuratie van de testen .De uitdaging hierbij zat hem in de tijd die nodig was om de testen op punt te stellen. Over de configuratieparameters is verschillende keren geïtereerd en elke iteratie kost veel tijd. Deze iteraties waren nodig bij het veranderen van de infrastructuur of de configuratie. Beiden kunnen als gevolg hebben dat het nieuwe systeem sneller of trager is. In de uiteindelijke configuratie duurt het uitvoeren van een enkele kalibratietest ongeveer 15 minuten, bij een consistentietest is dit ongeveer 10 minuten, bij een beschikbaarheidstest loopt dit op tot 20 minuten. 

\section{Verder werk}
In deze thesistekst zijn de eerste resultaten en conclusies naar beschikbaarheid en consistentie getrokken voor drie verschillende DBMS's. Een uitbreiding van het aantal geteste systemen zorgt er allereerst voor dat meer vergelijkend materiaal is maar daarnaast kunnen ook categorieën gevormd worden zoals bij het datamodel. 

Daarnaast kunnen de gebruikte testparameters ook aangepast worden om bepaalde assumpties te verifiëren of mathematische verbanden te zoeken. In de uitgevoerde testen hadden al de verschillende servers een ping tijd rond de 0.4ms, maar wat is bijvoorbeeld de invloed van deze parameter in de testen, hetzelfde geldt voor het aantal instanties van het DBMS en de belasting op de systemen (verkleint of vergroot het inconsistentie interval bij een hogere belasting?). 

Daarnaast kunnen ook de testmethode aangepast worden zoals het fysiek scheiden van de lezers en schrijven bij de consistentietest, wat een mogelijke invloed kan hebben op de resultaten. De beschikbaarheidstesten kunnen ook getest worden met verschillende fysieke gebruikers en onderzocht worden of deze hetzelfde gedrag hebben. 

Als laatste mogelijke uitbreiding, kunnen beide testen gecombineerd worden: verdwijnt er data als een instantie crasht en dit zowel vanuit het perspectief van de schrijver als de lezen. In MongoDB zou het mogelijk kunnen zijn dat een schrijfbewerking nog niet gerepliceerd was naar een secondary maar al wel gelezen was op de primary. Komt dit voor of zijn er mechanismen die dit voorkomen?  

\section{Evaluatie van de doelstellingen en bijdragen}
Deze thesis heeft met behulp van drie doelstellingen de kloof tussen de benodigde en beschikbare informatie kleiner gemaakt in relatie met de consistentie en beschikbaarheid van een DBMS. De verschillende doelstellingen en hun bijdrage zullen in deze sectie overlopen worden. 

\paragraph{Ontwikkelen van een benchmarking tool} In hoofdstuk \ref{sec:methodiekvantesten} is de algemene testmethode beschreven om de beschikbaarheid en consistentie van DBMS's te onderzoeken. Dit gebeurt door middel van verschillende stappen die bestaan uit het opstellen van de infrastructuur, uitvoeren van de testen en het analyseren van de resultaten. Hoofdstuk \ref{sec:implementatie} bespreekt hoe de verschillende stappen in de praktijk worden geïmplementeerd. De benchmarking tool is een uitbreiding van YCSB\cite{cooper2010benchmarking} met (1) ondersteuning voor het uitvoeren van instructies op gegeven tijdstippen en (2) een implementatie om leesbewerkingen uit te voeren tijdens en onmiddellijk na het schrijven van data. \\
De eerste uitbreiding is gebruikt voor de beschikbaarheidstesten, de tweede voor de consistentietesten. Deze benchmarking tool biedt basistesten aan die conclusies mogelijk maken. Zoals vermeldt in het verder werk kan deze testmethode verder uitgebreid worden. 

\paragraph{Analyseren van verschillende DBMS's} In hoofdstuk \ref{sec:observaties} zijn de observaties van drie verschillende DBMS's aan bod gekomen. Alle systemen zijn getest met behulp van de beschikbaarheidstesten, HBase en MongoDB zijn ook getest met de consistentietesten. In hoofdstuk \ref{sec:analyse} zijn de resultaten bestudeerd en de verschillen tussen de systemen besproken. \\
Uit de beschikbaarheidstesten blijkt dat de gecentraliseerde aanpak van Pgpool-II keer op keer dezelfde reactie geeft. De reacties van HBase en MongoDB zijn verschillend tussen verschillende uitvoeringen van dezelfde test. Beide systemen ondersteunen, in tegenstelling tot Pgpool-II, automatisch herstel van een node in het gedistribueerde opslag systeem. \\
Bij de consistentietesten is er een verschil hoe gelijktijdige bewerkingen worden uitgevoerd in MongoDB en HBase. Bij HBase wordt de schrijflock pas vrijgegeven na het volledig voltooien van de schrijfbewerking, hierna kan de leesbewerking uitgevoerd worden. In MongoDB wordt de schrijflock al vrijgegeven voor het volledig voltooien van de bewerking, met als gevolg dat leesacties de nieuwe data al lezen vooraleer een schrijfactie volledig voltooid is. \\
Bepaalde reacties van de systemen kunnen nog niet verklaard worden, maar met meer onderzoek is het mogelijk dat de reden achterhaald zou kunnen worden. Maar voor toekomstige gebruikers van deze DBMS's is er nu een overzicht wat de eigenschappen en het gedrag is bij het uitvallen van een service, node in het gedistribueerde opslag systeem of netwerk verbinding en hoe gelijktijdige lees- en schrijfbewerkingen afgehandeld worden. 

\paragraph{Eenvoudig herhalen en uitbreiden van de testen} Om de testen eenvoudig te kunnen herhalen, is de volledige testinfrastructuur op te stellen met behulp van IMP. Er is geen kennis nodig van de database systemen of de testsoftware, enkel de installatie van IMP met de nodige modules is nodig, gevolgd door het opstellen van de gewenste staat in een configuratie bestand en uiteindelijk de uitrol d.m.v. IMP. \\
De testen kunnen uitgebreid worden naar andere DBMS's, dit gebeurt in eerste instantie door de ondersteuning in YCSB te controleren of te implementeren, er is al standaard ondersteuning voor vele DBMS's. Daarna dienen de testen geconfigureerd worden specifiek voor deze database met behulp van de kalibratie. Daarna kunnen de beschikbaarheids- en consistentietesten voor dit nieuw systeem uitgevoerd worden.  
%\chapter{Conclusie}
In deze thesis is er gefocust op drie verschillende doelen: het creëren van een benchmark, het uitvoeren van de testen en het automatiseren van de testen. Elk van deze verschillende doelen zijn uitgewerkt waarbij de nodige problemen zijn overwonnen. 

Voor het implementeren van de benchmarkt werd een testmethode besproken en uitgewerkt, dit met het doel om de consistentie- en beschikbaarheidsverschillen van op een analytische manier te testen. Er is getest naar de beschikbaarheid van de systemen bij het verwachte en onverwacht stopzetten van instantie of netwerk onderbreking. Bij consistentie is er gekeken hoe lang het duurt voor de data beschikbaar is voor de verschillende gebruikers en welke garanties er aangeleverd kunnen worden. 

Vervolgens is de benchmark uitgevoerd op  HBase en MongoDB, een respectievelijk column en document database management systeem. Voor Pgpool-II, een gedistribueerde uitbreiding van PostgreSQL, zijn enkel de beschikbaarheidstesten uitgevoerd. 

De testresultaten tonen dat hoewel op papier HBase en MongoDB een gelijke strikte consistentie aanbieden, zijn er in de praktijk verschillen. HBase stelt de data beschikbaar voor alle leesgebruikers na de voltooiing van de schrijfbewerking, in tussentijd zullen de leesbewerkingen voor dat record vertraagd worden. MongoDB kiest ervoor om zo snel een leesbewerking te voltooien met indien mogelijk de nieuwe waarde, ook als de schrijfactie nog niet voltooid is. 

Bij de beschikbaarheidstesten is er groot verschil tussen de werking van Pgpool-II en de andere 2 systemen. Pgpool-II zal de status controleren door tussen de toegangsserver en de verschillende data instanties een data verbinding op te zetten. Indien deze verbinding verbroken wordt,is het datasysteem offline. 

HBase werkt met sessies van configureerbare duur, tijdens een sessie is een bepaalde server verantwoordelijk voor een deel van de data. Bij een verwachte stop kan deze sessie stopgezet worden en zal de data kort onbereikbaar zijn, bij een onverwachte stop of netwerk onderbreking wordt er gewacht tot na het verlopen van de sessie. Bij een harde stop kan er af en toe voorkomen dat er geen data gelezen wordt als men als gebruiker niet expliciet nieuwe connecties laat aanmaken, bij de andere mogelijkheden gebeurt dit automatisch. 

MongoDB werkt met een heartbeat protocol om de status van andere servers te controleren. Bij een verachte of onverwachte stop, is de data kort onbeschikbaar doordat een nieuwe verantwoordelijke moet worden aangeduid. Bij een netwerk onderbreking is, in tegenstelling tot het stoppen van een query, een nieuwe connectie met MongoDB nodig, anders zal de oude data niet gelezen worden. 

Als laatste doelstelling is de benchmarking software geautomatiseerd te samen met de drie database systemen voor de installeren en configuratie in een gedistribueerd systeem. De drie database systemen bestaan elk uit verschillende services en afhankelijkheden waarmee elk systeem de nodige uitdagingen met zich meebracht. Bij MongoDB is de installatie eenvoudig,  maar voor een automatische configuratie dient er code geschreven om de console configuratie te overkomen. HBase bestaat uit de combinatie van 3 verschillende software pakketten, die elk één voor één opgezet moeten worden. Pgpool-II heeft een zeer beperkte documentatie die voornamelijk gefocust is op Ubuntu, niet Fedora. 
\section{Verder werk}
In deze thesistekst zijn de eerste resultaten en conclusies naar beschikbaarheid en consistentie getrokken. Maar deze test methodes kunnen op meer systemen uitgevoerd worden tot een nieuwe vergelijkingsmethode voor vele database systemen.

Daarnaast kunnen de gebruikte testparameters ook aangepast worden om bepaalde assumpties te verifiëren of mathematische verbanden te zoeken. In de uitgevoerde testen hadden al de verschillende servers met een ping tijd rond de 0.5ms, maar wat is bijvoorbeeld de invloed van deze parameter in de testen, hetzelfde geldt voor het aantal instanties van het DBMS en de belasting op de systemen (verkleint of vergroot het inconsistentie interval bij een hogere belasting?). 

Daarnaast kunnen ook de testmethode aangepast worden zoals bij de consistentie test de lezer en schrijver fysiek scheiden. De beschikbaarheidstesten kunnen ook getest worden met verschillende fysieke gebruikers en te onderzoeken of deze hetzelfde gedrag meten. 

Als laatste mogelijke uitbreiding, kunnen beide testen gecombineerd worden: verdwijnt er data als een instantie crasht en dit zowel vanuit het perspectief van de schrijver als de lezen. In MongoDB zou het mogelijk kunnen zijn dat een schrijfbewerking nog niet gerepliceerd was naar een secondary maar al wel gelezen was op de primary. Komt dit voor of zijn er mechanismen die dit voorkomen?  



% Indien er bijlagen zijn:
\appendixpage*          % indien gewenst
\appendix
\chapter{Bespreking van verschillende DBMS's}\label{sec:BesprekingDBMS}

\begin{itemize}
\item Column NoSQL DBMS's: Cassandra, HBase
\item Document NoSQL DBMS's: Apache CoucheDB, MongoDB
\item Key-Value NoSQL DBMS's: LightCloud (Tokyo), MemCache, Redis, Riak, Project Voldemort
\item Relationele DBMS's: MySQL, Pgpool-II (PostgreSQL)
\end{itemize}

Deze keuze van deze systemen is gebaseerd op de paper van Christophe Strauch \cite{Strauch.NoSQL}. Een korte bespreking van de verschillende systemen kan gevonden worden in appendix 
\section{Column database}
\subsection{Cassandra}
\textit{Website: \url{http://cassandra.apache.org/}}\\
Cassandra is een database systeem die gebaseerd is op 2 verschillende systemen, Amazon's Dynamo en Google's Bigtable, wat voor een combinatie van een column- en key-value-based database zorgt. 

De query taal is beperkt tot 3 operaties: get, insert en delete \cite{Lakshman:2010:CDS:1773912.1773922}, waar de  laatste waarde in geval van een conflict zal opgeslagen worden.

De database kan gedistribueerd uitgerold worden waar door middel van partitionering en een consistent hashing algoritme de data verspreid wordt over de verschillende instanties. Om beschikbaarheid van de data te hebben bij een failure, wordt deze gerepliceerd over verschillende instanties met verschillende configuratie modellen. 

\subsection{HBase}
\textit{Website: \url{http://hbase.apache.org/}}\\
HBase is een database systeem die gebaseerd is op Google's BigTable en draait boven op HDFS, Hadoop Distibuted File System.

De query taal voor HBase bestaat uit 4 elementen, een get, put en delete als standaard operaties en een scan om over verschillende rijen te gaan. 

Voor het gedistribueerd draaien van de database, wordt de database ingedeeld in Regions. Vervolgens is een RegionServer verantwoordelijk voor de data van Regions. Daarnaast zijn er nog Zookeeper en Hadoop die respectievelijk verantwoordelijk zijn voor het management van de instanties en de eigenlijke dataopslag.

\section{Document database}
\subsection{Apache CoucheDB}
\textit{Website: \url{http://couchdb.apache.org/}}\\
Apache CoucheDB is een document database systeem waar alles wordt voorgesteld met behulp van JSON. Het systeem kan bevraagd worden door middel van Map-Reduce, de map gebeurd door een \textit{view}, een JavaScript-functie die de gegevens zal selecteren. Nadien kan met een reduce view de data geaggregeerd worden. 

Bij het gedistribueerd uitrollen zal de data met consistent hashing over verschillende instanties verdeeld worden waar elke instantie dezelfde rol heeft. Nu zal CoucheDB enkel updates van data van instantie veranderen en niet data automatisch verdelen. Ook is het mogelijk om een exacte replica van de ene naar de andere instantie te sturen, dit wordt bijvoorbeeld handig indien documenten naar een laptop gesynchroniseerd worden om later offline te kunnen werken.

In een gedistribueerde omgeving ziet CouchDB conflicten niet als een uitzondering maar als een normale omstandigheid. Wel zullen updates atomisch per rij afgewerkt worden op een enkele instantie, zodat hier geen conflict in kan bestaan. Maar indien een conflict optreedt, is het aan de bovenliggende applicatie om deze af te handelen. 

\subsection{MongoDB}
\textit{Website: \url{http://www.mongodb.org/}}\\
MongoDB is een document database systeem waar de data wordt voorgesteld aan de hand van BSON, een binaire vorm vergelijkbaar met JSON. Data kan ingegegeven worden via JSON aangezien er een eenvoudige map mogelijk is. 

Er is een uitgebreide query taal, waar er naast het invoegen, verwijderen en opvragen van een document ook talrijke zoekparameters meegegeven kunnen worden: dit gaat van zoeken op een enkel veld tot conjuncties, sorteren, projecties, ... 

MongoDB kan in een gedistribueerde omgeving opgezet worden met een opsplitsing tussen het redundant opslaan van data en het verdelen van data. Het redundant opslaan wordt toepast door het combineren van instanties in een ReplicaSet waar er een master-slave configuratie is. Daarnaast kan data ook verdeeld worden over verschillende instanties of replica sets, dit kan door middel van het configureren van shards. 
Conflicts worden opgevangen door de master waar er telkens een meerderheid van de instanties nodig is om deze te verkiezen. 

\section{Key-Value database}
\subsection{LightCloud (Tokyo)}
\textit{Website: \url{http://opensource.plurk.com/LightCloud/}}\\
LightCloud is een gedistribueerde uitbreiding van Tokyo Tyrant. Tokyo Tyrant is op zijn beurt een uitbreiding op Tokyo Cabinet en voegt de mogelijkheid tot externe connecties aan Cabinet toe. Cabinet is het basis pakket. 

De query taal is gelimiteerd tot 5 operaties: get, put, delete, add en een iterator om over de keys te gaan. Met add wordt er data aan een bestaand element toegevoegd. 

LightCloud levert een gedistribueerde database met master-master synchronisatie. Met behulp van een consistent hashing algoritme en 2 hash rings, wordt de data verdeeld over verschillende instanties met de nodige redundantie. De eerste ring is verantwoordelijk voor de lookups oftewel het lokaliseren van de keys, de storage ring is verantwoordelijk voor het opslaan van de verschillende waarden. 

\subsection{MemCacheDB}\todo{Updaten}
\textit{Website: \url{http://memcachedb.org/}}\\
MemCacheDB is een veel gebruikt systeem waarin al de data in RAM geheugen wordt gehouden en alhoewel er ondersteuning is met behulp van MemCacheDB voor persistentie is deze database niet bedoeld voor persistente opslag. 

De query mogelijkheden zijn beperkt tot get, put en delete van een waarde. In het geval een key meerdere keren geschreven wordt, zal de laatste waarde teruggegeven worden. 

\subsection{Redis}
\textit{Website: \url{http://www.redis.io/}}\\
Redis is een key-value database met de mogelijkheid tot opslaan van complexe datastructuren zoals lijsten, sets en mappen. Naast de standaard instructies om een enkele waarde toe te voegen, zijn er specifieke commando's om operaties op de complexere objecten te doen. Redis biedt ook ondersteuning voor transacties en heeft deze de mogelijkheid tot expire, hierdoor zal een waarde automatisch vergeten worden na een meegegeven tijd. 

De database wordt volledig in geheugen geplaatst maar ondersteunt 2 soorten van persistentie, oftewel door middel van RDB, oftewel met een AOF log. Bij RDB worden er over tijd snapshots gemaakt van de database en weggeschreven op harde schijf. In het geval van AOF wordt elke schrijfoperaties weggeschreven en kan de database opgebouwd worden met behulp van deze lijst.

Tenslotte heeft Redis momenteel een relatief beperkte mogelijkheid tot een gedistribueerde database. Het is mogelijk om data over verschillende instanties te distribueren met behulp van sharding welke op voorhand gedefinieerd dient te worden en is er ook de mogelijkheid tot master-slave opstelling met automatische failure detection.
De laatste is nog wel in beta, al is het mogelijk om deze te gebruiken. Tenslotte is er in de toekomst meer ondersteuning op komst met behulp van Redis Cluster waar data automatisch verspreid wordt over verschillende instanties. 

\subsection{Riak}
\textit{Website: \url{http://basho.com/riak/}}\\
Riak is een key-value database met de mogelijkheid tot opslaan van strings, JSON en XML. Daarnaast heeft deze standaard operaties maar hier enkele uitbreidingen op gemaakt. Allereerst is het mogelijk om secundaire indexen te definiëren op de elementen, MapReduce toe te passen en een full-text search. 

Riak is gebouwd om gedistribueerd te draaien waar al de instanties evenwaardig zijn. Data wordt verdeeld over de verschillende instanties en elk element wordt standaard op 3 verschillende instanties opgeslagen. Indien een bepaalde instantie faalt, wordt dit met een gossiping algoritme verspreid over de verschillende instanties waarmee een naburige instantie overneemt. Daarnaast is er automatische recovery indien een instantie terug online komt. 

\subsection{Project Voldemort}
\textit{Website: \url{http://www.project-voldemort.com/}}\\
Project Voldemort is een key-value store met enkel 3 basis operaties: get, put en delete met de mogelijkheid voor als keys en values strings, serializable objecten, protocol buffers of raw byte arrays te gebruiken. 

Deze database ondersteunt verschillende modes van distributie. De opbouw bestaat uit verschillende lagen, elk met hun eigen gedefinieerde functie. Met behulp van deze lagen kan de ontwikkelaar extra functionaliteit toevoegen met behulp van een extra laag om de applicatie meer te finetunen naar zijn uitwerking. 
Data wordt verdeeld met behulp van consistent hashing over de verschillende servers, waarbij data verschillende keren wordt bijgehouden om ervoor te zorgen dat de data nog beschikbaar is in het geval van falen. 

\section{Relationele database}
\subsection{MySQL}
\textit{Website: \url{http://www.mysql.com/}}\\
MySQL is een relationele database waarin data kan voorgesteld worden in verschillende vormen, beginnend met een bool tot een blok tekst. Daarnaast zijn de query mogelijkheden uitgebreid. 

De uitbreiding van een gedistribueerd systeem is bij MySQL ingebouwd door middel van een Master-Slave configuratie. Als mysqlfailover een faal detecteert in één van de slaven, zal de database verder werken, bij het falen van de master zal een nieuwe master handmatig aangeduid moeten worden. Ook de recovery moet handmatig gestart worden, waarna indien gewenst de originele master opnieuw als master kan gezet worden (bv. omdat deze de krachtigste computer is). 

\subsection{Pgpool-II (PostgreSQL)}
\textit{Website: \url{http://www.pgpool.net/}}\\
PostgreSQL is een relationele database en heeft soortgelijke specificaties als MySQL op een enkele computer, verschillende soorten data kunnen voorgesteld worden met uitgebreide query mogelijkheden. 

Enkel als de database ook gedistribueerd moet uitgerold worden, is er een verschil. Bij PostgreSQL is er standaard geen ondersteuning hiervoor maar moet er op externe elementen vertrouwd worden. Er bestaan verschillende componenten soorten systemen, maar het meeste uitgebreide pakket is Pgpool-II. Deze ondersteund load-balancing, een vergelijking van de systemen kan gevonden worden op de wiki van PostgreSQL \cite{postgresql-clustering}. 

Pgpool-II heeft verschillende mode, zoals parallel mode waar de data verdeeld wordt over verschillende instanties of replicatie waar de data op meerdere instanties wordt opgeslagen zodat deze nog beschikbaar is bij het falen van een enkele instantie.
\chapter{Overzicht van gedetailleerde implementatie keuzes}
In dit hoofdstuk bevinden zich het deployment van de testomgeving en de verschillende configuratie mogelijkheden voor de testuitbreidingen en configuratie van de testen. 
 
\begin{figure}[htbf]
\centering
\subfigure[Deployment van HBase met 5 instanties. ]{\label{fig:HBase-deployment}\includegraphics[width=0.55\textwidth]{img/HBase-deployment}}
\subfigure[Deployment van Pgpool-II met 3 instanties. ]{\label{fig:pgpool-deployment}\includegraphics[width=0.35\textwidth]{img/Pgpool-II-deployment}}
\subfigure[Deployment van MongoDB met 6 instanties. ]{\label{fig:MongoDB-deployment}\includegraphics[width=0.65\textwidth]{img/MongoDB-deployment}}
\subfigure[Deployment van de testomgeving met 2 YCSB instanties. ]{\label{fig:YCSB-deployment}\includegraphics[width=0.25\textwidth]{img/YCSB-deployment}}
\caption{Deployment van de verschillende DBMS's en de testomgeving.}\label{fig:deployment-testomgeving}
\end{figure}

\begin{figure}[htbf]

\begin{minipage}[b]{0.4\textwidth}
		\begin{tabular}{l|l}
			\textbf{Naam} & \textbf{eenheid} \\ 
			\hline ID & String \\ 
			Starttijdstip & milliseconden \\ 
			Commando & String \\ 
		\end{tabular} 
	\captionof{table}{Configuratie van event support}
	\label{table:beschikbaarheidinput}
\end{minipage}
\hfill
\begin{minipage}[b]{0.5\textwidth}

\begin{tabular}{l|l}
\textbf{Naam} & \textbf{eenheid} \\ 
\hline ID & String \\ 
Starttijdstip & milliseconden \\ 
Duur van de actie & microseconden \\
Gestart? & Boolean \\
Beëindigd? & Boolean \\
Exit code & Integer 
\end{tabular}
\captionof{table}{Uitvoer van event support}
\label{table:beschikbaarheidoutput}
\end{minipage}
\end{figure}


\begin{table}[htbf]
		\begin{tabular}{L{4cm}|L{2cm}|L{6.5cm}}
			\textbf{Naam} & \textbf{eenheid} & \textbf{Omschrijving} \\ 
			\hline 
			\parbox[t]{4cm}{consistencyTest} & Boolean & \parbox[t]{6.5cm}{Het activeren van de consistentie test}\\ 
			\parbox[t]{4cm}{addSeparateWorkload} & Boolean & \parbox[t]{6.5cm}{Het toevoegen van een basis belasting} \\ 
			\parbox[t]{4cm}{starttime} & \parbox[t]{2cm}{Milli-\newline seconden} & \parbox[t]{6.5cm}{Het startmoment van de consistentie test} \\
			\parbox[t]{4cm}{readThreads} & Integer & \parbox[t]{6.5cm}{Het aantal lees gebruikers} \\ 
			\parbox[t]{4cm}{consistencyDelayMillis} & \parbox[t]{2cm}{Milli-\newline seconden} & \parbox[t]{6.5cm}{Het interval waarin een lees gebruiker opnieuw het record leest} \\ 
			\parbox[t]{4cm}{newrequestperiodMillis} & \parbox[t]{2cm}{Milli-\newline seconden} & \parbox[t]{6.5cm}{Het interval waarin een schrijf gebruiker opnieuw een record schrijft} \\ 
			\parbox[t]{4cm}{insertProportion- \newline ConsistencyCheck} & \parbox[t]{2cm}{Float \newline ($0\leq x \leq 1)$} & \parbox[t]{6.5cm}{Het percentage van schrijfacties dat een nieuw record invoegt} \\ 
			\parbox[t]{4cm}{updateProportion- \newline ConsistencyCheck} & \parbox[t]{2cm}{Float \newline ($0\leq x \leq 1)$} & \parbox[t]{6.5cm}{Het percentage van schrijfacties dat een record aanpast} \\ 
			\parbox[t]{4cm}{stopOnFirstConsistency} & Boolean & \parbox[t]{6.5cm}{Stop zodra de eerste keer een correct record is gelezen} \\ 
			\parbox[t]{4cm}{maxDelayConsistency- \newline BeforeDropInMicros} & \parbox[t]{2cm}{Micro-\newline seconden} & \parbox[t]{6.5cm}{De maximale afwijking tussen de eigenlijke start van de query en het geplande moment} \\ 
			\parbox[t]{4cm}{timeoutConsistency- \newline BeforeDropInMicro} & \parbox[t]{2cm}{Micro-\newline seconden} & \parbox[t]{6.5cm}{De maximale tijd dat een leesactie geprobeerd wordt}\\
		\end{tabular} 
	\captionof{table}{Configuratie van de consistentie testen}
	\label{table:consistentieinput}
\end{table}

\begin{table}[htbf]
\centering
		\begin{tabular}{l|l|l}
			\textbf{Naam} & \textbf{eenheid} & \textbf{Omschrijving} \\ 
			\hline Tijd & Microseconden & Het moment dat de schrijfactie moest starten\\ 
			GebruikersID & R/W-Integer & Het id van de gebruiker (W-0, R-0, R-1, ..) \\ 
			Start & Microseconden & Het moment dat actie is begonnen \\
			Vertraging & Microseconden & De tijd dat de actie heeft geduurd \\ 
			Waarde & String & De gelezen of geschreven waarde \\ 
		\end{tabular} 
	\captionof{table}{Uitvoer van een enkel query in de consistentie testen}
	\label{table:consistentieuitvoer}
\end{table}

\begin{table}[htbf]
	\centering
		\begin{tabular}{l|l}
			\multicolumn{2}{c}{\textbf{Stoppen}} \\
			\textbf{Wat} & \textbf{Commando} \\ 
			\hline
			Zachte stop & service \{\{service-name\}\} stop \\ 
			Harde stop & kill -KILL \{\{process Id\}\} \\ 
			Netwerk onderbreken & iptables -A OUTPUT -d 0.0.0.0/0 -j DROP  \\ 
			\multicolumn{2}{c}{~} \\
			\multicolumn{2}{c}{\textbf{Heropstarten}} \\
			\textbf{Wat} & \textbf{Commando} \\ 
			\hline
			Zachte start & service \{\{service-name\}\} restart \\ 
			Harde start & service \{\{service-name\}\} restart \\ 
			Netwerk herstellen & iptables -D OUTPUT 1  \\ 
			\multicolumn{2}{c}{~}  \\
			\multicolumn{2}{c}{\textbf{Speciale commando's}} \\
			\textbf{Wat} & \textbf{Commando} \\ 
			\hline
			Pgpool-II & /usr/local/bin/pcp\_recovery\_node -d 10 \\
			\hspace*{0.5cm} (Online recovery) &  \hspace*{0.5cm} \{\{pgpool host\}\} \{\{port\}\} \{\{gebruikersnaam\}\} \\
			& \hspace*{0.5cm} \{\{wachtwoord\}\}  \{\{node nummer\}\}
		\end{tabular} 
	\captionof{table}{Beschikbaarheidstesten: Overzicht van de commando's voor het stoppen en starten in de verschillende modes. }
	\label{table:beschikbaarheidstesten-commandos}
\end{table}
\chapter{Extern beschikbare code en resultaten}\label{app:externlinks}
Dit hoofdstuk bevat een overzicht van de locaties van de externe code en resultaten. 
\begin{itemize}
	\item \textbf{IMP HBase} (\url{https://github.com/thuys/HBase}): De installatie en configuratie van HBase met behulp van IMP. 
	\item \textbf{IMP MongoDB} (\url{https://github.com/thuys/mongodb}): De installatie en configuratie van MongoDB met behulp van IMP. 
	\item \textbf{IMP Pgpool-II} (\url{https://github.com/thuys/postgresql}): De installatie en configuratie van Pgpool-II (PostgreSQL) met behulp van IMP. 
	\item \textbf{IMP YCSB} (\url{https://github.com/thuys/ycsb}): De installatie en configuratie van YCSB met behulp van IMP. 
	\item \textbf{YCSB code} (\url{https://github.com/thuys/YCSB-Implementation}): De aangepaste code van YCSB. 
	\item \textbf{YCSB Analyse code} (\url{https://github.com/thuys/YCSB-R-Scripts}): De R code om de YCSB analyse op uit te voeren. 
	\item \textbf{Ruwe testdata} (\url{https://github.com/thuys/YCSB-Testdata}): De ruwe testdata met grafieken gebruikt voor de thesis. 
\end{itemize}
%\chapter{Figuren van de observaties}\label{sec:figobservaties}
Dit hoofdstuk bevat de testdata op een grafische manier voorgesteld. Dit is enkel een selectie van de figuren, al de data en figuren kunnen gevonden worden op \url{https://github.com/thuys/YCSB-Testdata}.  

%\begin{figure}[ht!] 
%	\centering
%	\subfigure[Normal Update]{\label{fig:consistentie-all-mongodb-normal} \includegraphics[width=.40\textwidth]{img/Observaties/MongoDB/ECDF-Reads-update-normal-all-2}}
%	\subfigure[Safe Update]{\label{fig:consistentie-all-mongodb-safe} \includegraphics[width=.40\textwidth]{img/Observaties/MongoDB/ECDF-Reads-update-safe-all-2}}
%	\subfigure[Fsync Safe Update]{\label{fig:consistentie-all-mongodb-fsync} \includegraphics[width=.42\textwidth]{img/Observaties/MongoDB/ECDF-Reads-update-fsync_safe-all-2}}
%	\subfigure[Replica Safe Update]{\label{fig:consistentie-all-mongodb-replicasafe} \includegraphics[width=.40\textwidth]{img/Observaties/MongoDB/ECDF-Reads-update-replicas_safe-all-2}}
%	\subfigure[Majority Update]{\label{fig:consistentie-mongodb-all-majority} \includegraphics[width=.40\textwidth]{img/Observaties/MongoDB/ECDF-Reads-update-majority-all-2}}
%	\subfigure[Majority Insert]{\label{fig:consistentie-mongodb-all-majority-insert} \includegraphics[width=.40\textwidth]{img/Observaties/MongoDB/ECDF-Reads-insert-majority-all-2}}
%	\caption{Consistentie: Overzicht van MongoDB op de consistentie testen voor alle lezers gecombineerd met een 97-percentiel (voor de lezers) met start en stoptijden in dezelfde kleur. }
%	\label{fig:consistentie-mongodb-all}
%\end{figure}

%\begin{figure}[ht!] 
%	\centering
%	\subfigure[Normal Update]{\label{fig:consistentie-mongodb-R2-normal} \includegraphics[width=.40\textwidth]{img/Observaties/MongoDB/ECDF-Reads-update-normal-1-2}}
%	\subfigure[Safe Update]{\label{fig:consistentie-mongodb-R2-safe} \includegraphics[width=.40\textwidth]{img/Observaties/MongoDB/ECDF-Reads-update-safe-1-2}}
%	\subfigure[Fsync Safe Update]{\label{fig:consistentie-mongodb-R2-fsync} \includegraphics[width=.40\textwidth]{img/Observaties/MongoDB/ECDF-Reads-update-fsync_safe-1-2}}
%	\subfigure[Replica Safe Update]{\label{fig:consistentie-mongodb-R2-replicasafe} \includegraphics[width=.40\textwidth]{img/Observaties/MongoDB/ECDF-Reads-update-replicas_safe-1-2}}
%	\subfigure[Majority Update]{\label{fig:consistentie-mongodb-R2-majority} \includegraphics[width=.40\textwidth]{img/Observaties/MongoDB/ECDF-Reads-update-majority-1-2}}
%	\subfigure[Majority Insert]{\label{fig:consistentie-mongodb-R2-majority-insert} \includegraphics[width=.40\textwidth]{img/Observaties/MongoDB/ECDF-Reads-insert-majority-1-2}}
%	\caption{Consistentie: Overzicht van MongoDB op de consistentie testen voor lezer 2 met een 99-percentiel (voor de lezers) met start en stoptijden in dezelfde kleur.}
%	\label{fig:consistentie-mongodb-R2}
%\end{figure}

%\chapter{Uitwerking IMP} \label{chap:AppendixUitwerkingIMP}
In dit hoofdstuk zal voor de verschillende systemen de automatisering van installatie en configuratie met behulp van IMP uitgelegd worden. Er zal steeds de afhankelijkheden gegeven worden, een domeinmodel, uitleg bij het domeinmodel en voorbeeld configuratie gegeven worden. 

De automatisatie van installatie is ontwikkeld en getest met Fedora 18 en 20, op andere distributies en versies is er niet getest. 
Elk systeem maakt gebruik van \textit{ip::services::Server}, een instantie hiervan is een (virtuele) machine met een IP adres en besturingssysteem. 

Bij elke instantie is het verplicht om de firewall uit te zetten en SELinux op permissive te zetten. Dit kan met behulp van de volgende commando's: 
\begin{lstlisting}[frame=single, breaklines=true]
systemctl stop firewalld.service  
systemctl disable firewalld.service  
setenforce 0
sed -i "s/SELINUX=enforcing/SELINUX=permissive/g" /etc/sysconfig/selinux
sed -i "s/SELINUX=enforcing/SELINUX=permissive/g" /etc/selinux/config|
\end{lstlisting}

\section{HBase}
\textit{Link: \url{https://github.com/thuys/hbase}}

Benodigde IMP modules: std, net, ip, redhat, hosts en yum. 

De installatie en configuratie is gebeurd aan de hand van de uitleg en yum-repository van Cloudera\footnote{\url{http://www.cloudera.com/content/cloudera-content/cloudera-docs/CDH4/4.2.0/CDH4-Installation-Guide/CDH4-Installation-Guide.html}}. 

\subsection{Domein model en uitleg}
Het domeinmodel is te zien in figuur \ref{fig:imp-hbase-domeinmodel}.

\paragraph{HBaseMaster} Dit is de implementatie van de HMaster, dient toegewezen worden aan een host met java installatie. De poort is de poort waarop de HMaster actief is. 
 
\paragraph{HRegion} Dit is de implementatie van de HRegionServer, dient toegewezen worden aan een host met java installatie. De poort is de poort waarop de HRegionServer actief is. 

\paragraph{HadoopHDFS} Dit is de implementatie van de HDFS namenode, dient toegewezen worden aan een host met java installatie. De poort is de poort waarop de namenode actief is, de directory is de directory voor HBase  en de nameDir de lokatie waar de data op harde schijf weggeschreven zal worden. 

\paragraph{HadooDatapHDFS} Dit is de implementatie van de HDFS datanode, dient toegewezen worden aan een host met java installatie. De poort is de poort waarop de namenode datanode is, de directory is de directory voor HBase en de dataDir de lokatie waar de data op harde schijf weggeschreven zal worden. 

\paragraph{Zookeeper} Dit is de implementatie van een enkele Zookeeper. Bij het toewijzen van meerdere aan een cluster zullen de Zookeepers een cluster vormen. 

\paragraph{Javahost} Dit is een server waar Java is op geïnstalleerd. 

\begin{figure}[ht!]
\centering
\includegraphics[width=\linewidth]{img/HBase-Domeinmodel.png}
\caption{HBase: Domeinmodel HBase in IMP}
\label{fig:imp-hbase-domeinmodel}
\end{figure}

\subsection{Voorbeeld configuratie}
De configuratie voor de testomgeving gaat als volgt: 

\lstinputlisting[language=Python, breaklines=true, frame=single]{code/imp-hbase.conf}


\section{MongoDB}
\textit{Link: \url{https://github.com/thuys/mongodb}}

Benodigde IMP modules: std, net, ip, redhat, hosts en yum. 

De installatie en configuratie is gebeurd aan de hand van de uitleg en yum-repository van MongoDB\footnote{\url {http://docs.mongodb.org/manual/tutorial/install-mongodb- on-red-hat-centos-or-fedora-linux/}, \url{http://docs.mongodb.org/manual/tutorial/deploy-replica-set-for-testing/} en  \url{http://docs.mongodb.org/manual/tutorial/deploy-shard-cluster/}}. 

\subsection{Domein model en uitleg}
Het domeinmodel is te zien in figuur \ref{fig:imp-mongodb-domeinmodel}.

	\paragraph{MongoDB} is een server in het IMP model en is verantwoordelijk voor het installeren van de basis van MongoDB. Hierna zijn basis commando's voor connectie te maken met een MongoDB instantie beschikbaar. 
	\paragraph{MonogDBServer} is een server in het IMP model en is verantwoordelijk voor het installeren van de MongoDB server. 
	\paragraph{MongoDBNode} is de implementatie van een data instantie, maximaal 1 per server. Indien gelinkt met een replica set zal deze als een deel van een replica set worden geïnitialiseerd, anders als een zelfstandige instantie. 
	\paragraph{MongoDBReplicaSet} is de voorstelling van een replica set, dit wordt niet aan een specifieke server toegewezen. 
	\paragraph{MongoDBReplicaSetController} is verantwoordelijk om de replica set te initialiseren. Belangrijk is dat indien er een uitbreiding is van de set, de node verbonden met de controller een reeds geïnitialiseerde node is.  
	\paragraph{MongoDBConfigServer} is de implementatie van een configuratie server, 1 of 3 servers zijn nodig per cluster. 
	\paragraph{MongoDBAccessServer} is de implementatie van mongos, minstens 1 is nodig maar meer kunnen gebruikt worden.
	\paragraph{MongoDBShardCluster} is de voorstelling van een cluster van shards, er kunnen zowel alleenstaand instanties als replica sets aan toegevoegd worden. 
	\paragraph{MongoDBShardController} is verantwoordelijk om de cluster te initialiseren met de verschillende shards, databases, collecties en keys. 
	\paragraph{MongoDBDatabase} is de voorstelling van een database.
	\paragraph{MongoDBCollection} is de voorstelling van een collectie, indien verbonden met een cluster via een database zal deze gedeeld worden over de verschillende shards. 
	\paragraph{MongoDBKey} is de wijze waarmee een collectie verdeeld wordt over de verschillende shards. 

\begin{figure}[ht!]
\centering
\includegraphics[width=\linewidth]{img/MongoDB-Domeinmodel.png}
\caption{MongoDB: Domeinmodel MongoDB in IMP}
\label{fig:imp-mongodb-domeinmodel}
\end{figure}

\subsection{Voorbeeld configuratie}
De configuratie voor de testomgeving gaat als onderstaand. Bij de uitrol van IMP gaat dit verschillende keren uitgevoerd moeten worden omdat eerst de MongoDBNodes moeten draaien, vervolgens kunnen de replicasets aangemaakt worden, daarna kunnen de replicasets pas toegevoegd worden in de cluster. 

In IMP was het nog niet mogelijk om een te zeggen dat x uitgevoerd moet zijn op een andere instantie, vooraleer y kan uitgevoerd worden, ondertussen is dit mogelijk door de thesis van Harm De Weirdt\cite{thesisHarm} waar de nieuwe installatie beschikbaar is op \url{https://github.com/Foezjie/mongodb} maar hierbij dient ook gebruik gemaakt te worden van zijn IMP installatie.  \\
Met het ontbreken hieraan kan het zijn dat er 3 keer een volledige IMP deploy uitgevoerd moet worden. 

\lstinputlisting[language=Python, breaklines=true, frame=single]{code/imp-mongodb.conf}


\section{Pgpool-II}
\textit{Link: \url{https://github.com/thuys/postgresql}}

Benodigde IMP modules: std, net, ip, redhat, hosts en yum. 

De installatie en configuratie is gebeurd aan de hand van de uitleg Pgpool-II\footnote{\url {http://pgpool.projects.pgfoundry.org/pgpool-II/doc/tutorial-en.html/}}. 

\subsection{Domein model en uitleg}
Het domeinmodel is te zien in figuur \ref{fig:imp-pgpool-domeinmodel}.

	\paragraph{PgpoolMain} Dit is de implementatie van de Pgpool-II router node. 
	
	\paragraph{PgpoolNode} Dit is de implementatie van de Pgpool-II data node die een uitbreiding is van de standaard PostgreSQL installatie. 
		
	\paragraph{PostgresqlServer} Dit is de implementatie van de standalone PostgreSQL server. 

\begin{figure}[ht!]
\centering
\includegraphics[width=\linewidth]{img/Postgres-Domeinmodel.png}
\caption{Pgpool-II: Domeinmodel Pgpool-II in IMP}
\label{fig:imp-pgpool-domeinmodel}
\end{figure}

\subsection{Voorbeeld configuratie}
De configuratie van Pgpool-II gebeurt in verschillende stappen: shell code, IMP uitrol, extra configuratie stap, IMP uitrol. 

De eerste shell code bestaat erin om de SELinux volledig uit te schakelen: 
\begin{lstlisting}[frame=single, breaklines=true]
systemctl stop firewalld.service  
systemctl disable firewalld.service  
echo "SELINUX=disabled SELINUXTYPE=targeted" > /etc/selinux/config

echo "SELINUX=disabled SELINUXTYPE=targeted" > /etc/sysconfig/selinux
\end{lstlisting}

De configuratie voor de testomgeving gaat als volgt in IMP: 

\lstinputlisting[language=Python, breaklines=true, frame=single]{code/imp-pgpool.conf}

De configuratie bestaat erin om al de verschillende nodes van Pgpool-II, ongeachte of dit routers of datanodes zijn, ssh toegang te geven tot elkaar server via ssh met root en postgres als gebruikers. Deze verbinding al een keer gemaakt zijn want een bericht dat de sleutel nu mee is opgeslagen is voldoende om de online recovery te doen falen. 

Hierna kan de IMP uitrol nog een keer gebeuren en het systeem zou moeten werken. 


\section{YCSB}
\textit{Link: \url{https://github.com/thuys/ycsb}}

Benodigde IMP modules: std, net, ip, redhat, hosts, yum, git, hbase, mongodb, pgpool-II 

De installatie en configuratie is gebeurd aan de hand van de uitleg van YCSB\footnote{\url {https://github.com/brianfrankcooper/YCSB/wiki/}}. 

\subsection{Domein model en uitleg}
Het domeinmodel is te zien in figuur \ref{fig:imp-ycsb-domeinmodel}.
\begin{figure}[ht!]
\centering
\includegraphics[width=\linewidth]{img/YCSB-Domeinmodel.png}
\caption{YCSB: Domeinmodel YCSB in IMP}
\label{fig:imp-ycsb-domeinmodel}
\end{figure}

Dit model bevat maar 1 nieuw element en dit is YCSB. Deze dient verbonden te zijn met één van de 3 systemen die hierboven zijn beschreven. Elk van deze systemen zal getest worden voor alle testen die geactiveerd zijn. MongoDB heeft 2 connecties omdat de cluster voor de beschikbaarheidstesten wordt gebruikt en een replicaset voor de consistentie testen. Pgpool-II heeft geen ondersteuning voor de consistentie testen. 

\subsection{Voorbeeld configuratie}

De configuratie voor de testomgeving gaat als volgt in IMP: 

\lstinputlisting[language=Python, breaklines=true, frame=single]{code/imp-ycsb.conf}

De testen starten door het uitvoeren van {{directory}}/scripts/ycsb-script. De resultaten komen in de folder {{directory}}/results. 

\chapter{Paper}
\includepdf[pages={1,2,3}]{paper/paper.pdf}
\chapter{Poster}\label{sec:poster}
De poster met als titel '\textit{CAP in de praktijk: MongoDB}' bevindt zich op de volgende pagina. 
\includepdf[pages={1}, scale=0.8, frame]{poster/poster.pdf}

%\include{appendix/app-A}
% ... en zo verder tot
%\include{appendix/app-n}

\backmatter
% Na de bijlagen plaatst men nog de bibliografie.
% Je kan de  standaard "abbrv" bibliografiestijl vervangen door een andere.
%\printglossaries

\printbibliography


\end{document}

%%% Local Variables: 
%%% mode: latex
%%% TeX-master: t
%%% End: 
