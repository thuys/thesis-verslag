\chapter{Observaties}
In dit hoofdstuk worden de resultaten getoond van de testen die zijn uitgevoerd, met een bespreking van de speciale elementen. Er zullen nog geen redenen aangehaald of conclusies getrokken worden, dit wordt in het volgende hoofdstuk gedaan. 

De resultaten zullen besproken worden per testsoort: eerst de resultaten voor de calibratie daarna voor beschikbaarheid en tenslotte voor consistentie. De figuren bevinden zich in bijlage \ref{sec:figobservaties}

De ruwe testdata kan geraadpleegd worden op \url{https://github.com/thuys/YCSB-Testdata}. 

\section{Calibratie}

\paragraph{Aantal gebruikers}
De resultaten van de calibratietest voor het aantal gebruikers kunnen gevonden worden in figuur \ref{fig:calibratie-gebruikers-resultaat}. Op de x-as is het gemiddeld aantal queries per second getoond over een periode van 600s, op de y-as de gemiddelde vertraging. De verschillende punten stellen een aantal gebruikers voor die zijn aangegeven met het bijhorend getal. 

Het aantal gebruikers wordt zo gekozen dat het totale aantal queries zakt of voor een sterke groei in vertraging zorgt, dit zorgt voor de gegevens in tabel \ref{table:calibratie-gebruikers-resultaat}. Bij MongoDB is er voor een lage waarde van 15 gebruikers gekozen, in plaats van 50, de reden hiervoor is dat de variatie in de vertraging groter wordt bij meer gebruikers, wat de data van de overige testen potentieel minder stabiel maakt.  

\begin{table}[h!]
	\centering
	\begin{tabular}{l| l }
		\textbf{DBMS} & Aantal gebruikers \\
		\hline
		HBase & 50 \\
		MongoDB & 15\\
		Pgpool-II & 30\\
	\end{tabular}
	\caption{Calibratie: Aantal gebruikers per test voor de verschillende DBMS's}
	\label{table:calibratie-gebruikers-resultaat}
\end{table}

\paragraph{Aantal queries per seconde}
De resultaten voor de calibratietest voor het aantal queries per seconden kunnen gevonden worden in de figuren \ref{fig:calibratie-queriesperseconde-hbase}, \ref{fig:calibratie-queriesperseconde-mongodb} en \ref{fig:calibratie-queriesperseconde-pgpool-ii} voor respectievelijk HBase, MongoDB en Pgpool-II. Deze figuren tonen in de bovenste figuur de gemiddelde vertraging op een query afhankelijk van het aantal queries per seconde, naarmate het aantal queries toeneemt stijgt de vertraging, met uitzondering van een laag aantal queries. De onderste figuur toont op de y-as de verhouding tussen het eigenlijk aantal uitgevoerde queries per seconde t.o.v. het gevraagde aantal queries per seconde. Een fictief voorbeeld: bij het vragen van 100 queries/sec wordt er in de praktijk maar 60 uitgevoerd, dit zorgt voor een waarde van $0.6$. 

Met beide figuren samen, kan een matige belasting gekozen. Een matige belasting is een belasting waarbij de onderste figuur de waarde 1 zo dicht mogelijk benaderd en de vertraging nog niet te veel is gestegen t.o.v. van een lage belasting. De gekozen waarde zijn te vinden in tabel \ref{table:calibratie-queriesperseconde-resultaat}. 

\begin{table}[htb!]
	\centering
	\begin{tabular}{l| l }
		\textbf{DBMS} & Aantal requests per seconde \\
		\hline
		HBase & 600 \\
		MongoDB & 200\\
		Pgpool-II & 100\\
	\end{tabular}
	\caption{Calibratie: Aantal queries per seconde per test bij een matige belasting voor de verschillende DBMS's.}
	\label{table:calibratie-queriesperseconde-resultaat}
\end{table}



\section{Beschikbaarheidstest}
Bij de beschikbaarheidstesten kunnen de gegevens op verschillende manieren voorgesteld worden: de vertraging per query over de hele test, de vertraging tijdens het stoppen en starten van systemen of een vergelijking van de vertraging voor het stoppen (150-250s), na het herstarten (700-800s) en tussen het stoppen en starten(400-500s). 

Voor elk van de systemen is voor al de acties op de verschillende instanties data in voorhand, maar slechts enkele grafieken zullen getoond worden. Al de grafieken kunnen gevonden worden op GitHub op de link gegeven in het begin van het hoofdstuk. 

Een punt op de grafiek stelt de gemiddelde vertraging van 1 seconde voor, de lijn het gemiddelde over 10 seconden. 

\paragraph{HBase}
Bij HBase zijn er verschillende reacties op het stopzetten van een node. Bij het zacht stoppen van een instantie, is er een onderbreking van gemiddeld ongeveer 20 seconde in de testen. Daarna kunnen er nog verhogingen in de queries af en toe optreden. Zie figuur \ref{fig:beschikbaar-hbase-soft}. 

Bij de netwerk onderbreking is er in de testen een onderbreking van gemiddeld ongeveer 100 seconden. Daarna is het terug stabiel. Zie figuur \ref{fig:beschikbaar-hbase-drop}

Bij een harde stop is er een combinatie van de netwerk onderbreking én is het af en toe zo dat de volledige periode geen queries mogelijk zijn. Zie figuur \ref{fig:beschikbaar-hbase-drop} en \ref{fig:beschikbaar-hbase-hard}

Tijdens de onderbreking (400-500s), is er geen significante verandering in de vertraging van de uitgevoerde queries gemeten (t.o.v. 150-250s en 700s - 800s). 


\paragraph{MongoDB}
Ook bij MongoDB zijn er verschillende reacties op het stopzetten als men de queries onder standaard configuratie uitvoert. In het geval van zacht of hard stoppen is er geen verschil in de reactie, bij 2/3 van de keren is er geen verschil merkbaar, bij 1/3 van de keren is er tijdelijke verhoging van de vertraging, een voorbeeld toont dat de scan operatie voor 2 seconden uitgesteld wordt. Figuur van het overzicht: \ref{fig:beschikbaar-mongodb-soft} met een zoom naar de stop \ref{fig:beschikbaar-mongodb-soft-zoom}.

Bij het onderbreken van het netwerk is er in bepaalde geen significante verandering, op andere momenten is een gedrag soortgelijk aan dat bij een zachte stop te merken. In andere gevallen is het zo dat er geen queries mogelijk zijn gedurende de volledige netwerk onderbreking. Zie figuur \ref{fig:beschikbaar-mongodb-drop}. 

Tijdens de onderbreking (400s- 500s), is er geen significante verandering in de vertraging van de uitgevoerde queries gemeten (t.o.v. 150-250s en 700s - 800s). 
 


\paragraph{Pgpool-II} Bij Pgpool-II is een green verschil tussen een harde of zacht stop. In beide gevallen is er tijdelijk een onderbreking van al de queries, er is een verhoogde vertraging van ongeveer 2 seconden, een voorbeeld bevindt zich in figuur \ref{fig:beschikbaar-pgpool-soft}. 

Bij een netwerk onderbreking, zijn er enige tijd geen queries mogelijk en na 30 seconden was de onderbreking over in de testen. Een voorbeeld bevindt zich in figuur \ref{fig:beschikbaar-pgpool-netwerk}.  

Tijdens de onderbreking is er een verandering naar schrijfbewerkingen toe, deze nemen significant minder tijd in beslag, dit geldt niet voor leesbewerkingen. Voorbeelden uit de testen bevinden zich voor beiden in figuren \ref{fig:beschikbaar-pgpool-boxplot-read} en \ref{fig:beschikbaar-pgpool-boxplot-write}. 

De herstel van een server na deze opnieuw online gebracht hebben, lukt slechts in zeldzame keren na deze te hebben opstarten. Enkel als alle connecties zijn, verbroken lukt het herstel. 

\section{Consistentie test}
Voor de consistentie testen worden er empirische verdelingsfuncties gebruikt. Dit zijn functies waarbij op de y-as het percentage van de waarden kleiner dan x staat aangegeven.

Voor de consistentie testen wordt er op de x-as de start- en/of stoptijdstippen van de verschillende soorten getoond. Het verschil tussen de y-waarde van de start- en stoptijdstippen geeft aan hoeveel queries er op dat moment uitgevoerd worden. De startmomenten van een lezer zijn de eerste keer dat deze de correcte data leest. 

\paragraph{HBase}
Bij HBase is er geen verschil tussen het invoegen of aanpassen van data naar consistentie, dit zijn dezelfde queries. Daarnaast zijn er geen configuratie mogelijkheden voor het lezen of schrijven van data naast het in- of uitschakelen van de caches aan de gebruikerskant, deze uitgeschakeld zijn al de testen. 

Figuur \ref{fig:consistentie-hbase-start} toont een overzicht van de verschillende starttijdstippen voor het lezen van consistente data. Figuur \ref{fig:consistentie-hbase-R1} toont de start- en eindtijdstippen voor lezer 2 naast deze voor de schrijver. De maximale waarde van de x-as is zo gekozen dat voor elke dataset minstens 99\% van de data getoond is. 



\paragraph{MongoDB} Bij MongoDB zijn er 5 soorten lees- en 5 soorten schrijfconfiguraties mogelijk. Na de testen bleek het dat secondarypreferred gelijk was aan secondary en is niet getoond in de grafieken om deze reden. 

Voor elk van de 5 mogelijke schrijvacties, is een overzicht gegeven in figuren \ref{fig:consistentie-mongodb-all} en \ref{fig:consistentie-mongodb-R2}. In de eerste figuur, is de data van al de lezers gecombineerd, bij de tweede wordt er enkel naar lezer 2 gekeken. De maximale waarde van de x-as is zo gekozen dat voor elke dataset minstens 97\% en 99\% van de leesdata is getoond inor respectievelijk figuur \ref{fig:consistentie-mongodb-all} en \ref{fig:consistentie-mongodb-R2}. Er zijn telkens de update queries getoond met een vergelijking tussen de insert en update bij majority. Ook uit de andere data blijkt dat deze niet significant verschillend zijn. 

Een vergelijking van de duur van de verschillende leesoperaties kan gevonden worden in figuur \ref{fig:consistentie-mongodb-all-mongodb-write} waar er minstens 90\% van de leesdata is getoond. 


\section{Conclusie}
De resultaten zijn verschillend voor de verschillende systeem, in het volgend hoofdstuk zullen de resultaten geanalyseerd worden en mogelijke verklaringen gezocht worden. 
